\section{Organization}
A very important factor for successful thesis writing is the
organization of the material. This template suggests a structure as
the following:
\begin{itemize}
    \marginpar{You can use these margins for summaries of the text
    body\dots}
    \item\texttt{Chapters/} is where all the ``real'' content goes in
    separate files such as \texttt{Chapter01.tex} etc.
 %  \item\texttt{Examples/} is where you store all listings and other
 %  examples you want to use for your text.
    \item\texttt{FrontBackMatter/} is where all the stuff goes that
    surrounds the ``real'' content, such as the acknowledgments,
    dedication, etc.
    \item\texttt{gfx/} is where you put all the graphics you use in
    the thesis. Maybe they should be organized into subfolders
    depending on the chapter they are used in, if you have a lot of
    graphics.
    \item\texttt{Bibliography.bib}: the Bib\TeX\ database to organize
    all the references you might want to cite.
    \item\texttt{classicthesis.sty}: the style definition to get this
    awesome look and feel. Does not only work with this thesis template
    but also on its own (see folder \texttt{Examples}). Bonus: works
    with both \LaTeX\ and \textsc{pdf}\LaTeX\dots and \mLyX.
    \item\texttt{ClassicThesis.tcp} a \TeX nicCenter project file.
    Great tool and it's free!
    \item\texttt{ClassicThesis.tex}: the main file of your thesis
    where all gets bundled together.
    \item\texttt{classicthesis-config.tex}: a central place to load all 
    nifty packages that are used. %In there, you can also activate 
    %backrefs in order to have information in the bibliography about 
    %where a source was cited in the text (\ie, the page number).
    
    \emph{Make your changes and adjustments here.} This means that you  
    specify here the options you want to load \texttt{classicthesis.sty} 
    with. You also adjust the title of your thesis, your name, and all 
    similar information here. Refer to \autoref{sec:custom} for more 
    information.
    
        This had to change as of version 3.0 in order to enable an easy 
        transition from the ``basic'' style to \mLyX.
    
\end{itemize}
In total, this should get you started in no time.


\clearpage
\section{Style Options}\label{sec:options}
There are a couple of options for \texttt{classicthesis.sty} that
allow for a bit of freedom concerning the layout:
\marginpar{\dots or your supervisor might use the margins for some
    comments of her own while reading.}
\begin{itemize}
    \item General:
        \begin{itemize}
            \item\texttt{drafting}: prints the date and time at the bottom of
    each page, so you always know which version you are dealing with.
    Might come in handy not to give your Prof. that old draft.
        \end{itemize}
    
    \item Parts and Chapters:
        \begin{itemize}
            \item\texttt{parts}: if you use Part divisions for your document,
    you should choose this option. (Cannot be used together with 
    \texttt{nochapters}.)
    
            \item\texttt{nochapters}: allows to use the look-and-feel with 
    classes that do not use chapters, \eg, for articles. Automatically
    turns off a couple of other options: \texttt{eulerchapternumbers}, 
    \texttt{linedheaders}, \texttt{listsseparated}, and \texttt{parts}. 
    
        \item\texttt{linedheaders}: changes the look of the chapter
        headings a bit by adding a horizontal line above the chapter
        title. The chapter number will also be moved to the top of the
        page, above the chapter title.
    
        \end{itemize}

  \item Typography:
        \begin{itemize}
            \item\texttt{eulerchapternumbers}: use figures from Hermann Zapf's
            Euler math font for the chapter numbers. By default, old style
            figures from the Palatino font are used.
    
            \item\texttt{beramono}: loads Bera Mono as typewriter font. 
            (Default setting is using the standard CM typewriter font.)
            
            \item\texttt{eulermath}: loads the awesome Euler fonts for math. 
            Pala\-tino is used as default font.
    
            \item\texttt{pdfspacing}: makes use of pdftex' letter spacing
            capabilities via the \texttt{microtype} package.\footnote{Use 
            \texttt{microtype}'s \texttt{DVIoutput} option to generate
            DVI with pdftex.} This fixes some serious issues regarding 
            math formul\ae\ etc. (\eg, ``\ss'') in headers. 
            
            \item\texttt{minionprospacing}: uses the internal \texttt{textssc}
            command of the \texttt{MinionPro} package for letter spacing. This 
            automatically enables the \texttt{minionpro} option, overriding
            \texttt{pdfspacing}.
    
        \end{itemize}  

    \item Table of Contents:
        \begin{itemize}
             \item\texttt{tocaligned}: aligns the whole table of contents on
            the left side. Some people like that, some don't.
            
            \item\texttt{dottedtoc}: sets pagenumbers flushed right in the 
            table of contents.

            \item\texttt{manychapters}: if you need more than nine chapters for 
        your document, you might not be happy with the spacing between the 
        chapter number and the chapter title in the Table of Contents. 
        This option allows for additional space in this context. 
        However, it does not look as ``perfect'' if you use
        \verb|\parts| for structuring your document.
            
        \end{itemize}
    
    \item Floats:
        \begin{itemize}
    \item\texttt{listings}: loads the \texttt{listings} package (if not 
    already done) and configures the List of Listings accordingly.
    
    \item\texttt{floatperchapter}: activates numbering per chapter for
    all floats such as figures, tables, and listings (if used). 
    
        \item\texttt{subfig}(\texttt{ure}): is passed to the \texttt{tocloft} 
        package to enable compatibility with the \texttt{subfig}(\texttt{ure}) 
        package. Use this option if you want use \texttt{classicthesis} with the
        \texttt{subfig} package.
        
%    \item\texttt{listsseparated}: will add extra space between table
%    and figure entries of different chapters in the list of tables or
%    figures, respectively. % Deprecated as of version 2.9.
        \end{itemize}    
 
%   \item\texttt{a5paper}: adjusts the page layout according to the
%    global \texttt{a5paper} option (\emph{experimental} feature).
%    \item\texttt{minionpro}: sets Robert Slimbach's Minion as the 
%    main font of the document. The textblock size is adjusted 
%    accordingly.    

   \end{itemize}
The best way to figure these options out is to try the different
possibilities and see what you and your supervisor like best.

In order to make things easier, \texttt{classicthesis-config.tex} 
contains some useful commands that might help you.