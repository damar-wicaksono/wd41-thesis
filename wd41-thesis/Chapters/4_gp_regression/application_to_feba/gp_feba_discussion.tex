%****************************************************
\subsection{Discussion}\label{sub:gp_feba_discussion}
%****************************************************

The selection of number of \glspl[hyper=false]{pc} to retain is usually done by justifying the amount of total variance explained by the selected \glspl[hyper=false]{pc}.
This is also directly related to

\normdoublefigure[pos=tbhp,
                 mainlabel={fig:ch4_plot_hit_and_miss},
								 mainshortcaption={Principal Component Analysis of a bivariate dataset},
                 maincaption={\gls[hyper=false]{pca} of a bivariate dataset. A highly correlated bivariate dataset can be transformed into a new orthogonal coordinate system according to the principal direction of the dataset. The principal directions redistribute the total variance such that the partial variance is preserved in the transformed coordinate. Illustrated above, $5$ selected points in the dataset in both coordinates.},
                 leftopt={width=0.475\textwidth},
                 leftlabel={fig:ch4_plot_hit_and_miss_2},
                 leftcaption={Original data},
                 rightopt={width=0.475\textwidth},
                 rightlabel={fig:ch4_plot_hit_and_miss_1},
                 rightcaption={Transformed data}]
{../figures/chapter4/figures/plotHitAndMiss_1}
{../figures/chapter4/figures/plotHitAndMiss_2}
