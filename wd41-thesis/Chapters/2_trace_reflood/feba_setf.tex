\section{FEBA Reflood Separate Effect Test Facility}\label{sec:reflood_feba_setf}

A series of FEBA experiments was conducted in the 1980s at the Karlsruhe Institute of Technology\footnote{formerly Kernforschungzentrum Karlsruhe (KfK)}
to improve the knowledge of the heat transfer (\textsc{HT}) mechanism during reflooding,
taking into account the effects of spacer grids and flow blockage due to fuel rod ballooning.
The data from the facility was also intended to validate the TH models and codes available at the time.

The facility consisted of a test section with full height $5 \times 5$ bundle of \textsc{PWR} fuel rod simulator (Fig.~\ref{fig:feba_setf}a) enclosed in a rectangular stainless steel housing (Fig.~\ref{fig:feba_setf}b).
An approximate cosine power profile was mapped over of the height of the fuel rod simulators (Fig.~\ref{fig:feba_setf}c).
Seven spacer grids were used to provide mechanical support of the fuel rod simulators (Fig.~\ref{fig:feba_setf}d).

\begin{figure}[bth]
	\includegraphics[width=1.0\textwidth]{../figures/febaTestSection/febaTestSection.png}
	\caption[FEBA experimental facility]{(a) The cross section of the fuel rod simulators used in FEBA separate-effect test facility. (b) the cross section of the test section including the rectangular housing. (c)  The approximate cosine power profile, numbers written inside the bix are the relative power $P/P_{avg}$. (d) The location of spacer grids in the test section. All dimensions are in units of milimeters $[mm]$}\label{fig:feba_setf}
\end{figure}

During the initialization phase of the experiment, the test section was heated up at low nominal power ($200$~$[kW]$) to achieve a specified initial heater rod temperature, with no liquid present in the test section.
The transient phase of the experiment was initiated by ramping
up the power according to $120$\% American National Standard
decay heat power curve while simultaneously injecting
subcooled liquid from the bottom of the test section.
Several temperature measurements at the outer surface of
the heater rods, hereinafter referred to as the cladding
temperature, were taken at different axial locations during
the course of each transient test.

Eight different test series were performed in the
FEBA facility. 
The first two test series (I and II) used two
different numbers of spacer grids, seven and six, respectively.
The middle spacer grid was removed in test series II
to investigate the effect of spacer grids in a reflood transient.
The other test series used different flow area blockage sizes
at midheight of the test section to investigate the effect of rod
ballooning of different sizes. In each test series, combinations
of two different inlet liquid velocities and three different
system backpressures were imposed.

The present study analyzed only the experimental dataset from test series \textsc{I}.
The test run was used as the base experimental setup with all seven spacer grids mounted and no flow area blockage.
Different experimental runs corresponding to different experimental conditions of this particular test series are given in Table~.

