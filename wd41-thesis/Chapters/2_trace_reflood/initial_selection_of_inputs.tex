%******************************************************************************************
\section{Initial Selection of Input Parameters}\label{sec:reflood_initial_inputs_selection}
%******************************************************************************************

% Introductory paragraph
In this section, the selection of the initial set of input parameters of the \gls[hyper=false]{feba} simulation in \gls[hyper=false]{trace} is presented.
Afterward, the initial (prior) uncertainties of these parameters are presented. 
This part is closely related to the author's participation to the \gls[hyper=false]{premium} benchmark.

%---------------------------------------------------------------------------------
\subsection{Selection of Input Parameters}\label{sub:reflood_parameters_selection}
%---------------------------------------------------------------------------------

% Introductory paragraph
The selection process for the input parameters differs depending on the type of parameter.
Each of the selected parameters can broadly fall in of the two following categories:
\begin{itemize}
	\item Input parameters that are not specific to the \gls[hyper=false]{trace} code (e.g., initial and boundary conditions, material thermo-physical properties).
		This category of parameters is often referred to as the \emph{controllable inputs} of the simulation.
	\item Input parameters that are specific to \gls[hyper=false]{trace} code (e.g., implementation of the $2$-phase momentum and heat transfer package for reflood condition).
		This category is often referred to as the \emph{model parameters} of the simulation.
\end{itemize}

% Selection on the first category
The selection of parameters belonging to the controllable inputs category simply corresponds to the parameters recommended by the benchmark organizers and employed by most participants \cite{Kovtonyuk2015}.

% Selection on the second category
On the other hand, the selection of the model parameters specific to the \gls[hyper=false]{trace} code is challenging due to the fact that \gls[hyper=false]{trace} is a relatively recent code (in comparison with the codes like RELAP5, ATHLET, or CATHARE).
In essence, the code has been developed from different variants of the TRAC codes for different reactor types (TRAC-BF1, TRAC-P) to result in a single consolidated code applicable to both \gls[hyper=false]{pwr} and BWR\footnote{boiling water reactor}.
Contributing to that difficulty is the fact that \gls[hyper=false]{trace} is currently undergoing significant developments and improvements, including modifications to the $2$-phase closure models for momentum and heat transfers.
Consequently, the task of selecting the model parameters and later their prior uncertainties is more difficult than for a more established codes.

% Criteria for choosing the parameters

%-----------------------------------------------------------------------------------
\subsection{Prior Uncertainty Quantification}\label{sub:reflood_parameters_prior_uq}
%-----------------------------------------------------------------------------------

% Opening paragraph

The nominal values of selected model parameters of the \gls[hyper=false]{trace} \gls[hyper=false]{feba} model are varied by their respective perturbation factors.
These perturbation factors are modeled as random variables following a predefined \gls[hyper=false]{pdf}, 
from which a set of samples of model parameters values can be generated.
The choice of the distribution is influenced by both the available information and the aim of the analysis.
As the aim of this particular step is to obtain preliminary (or \emph{prior}) uncertainty of the \gls[hyper=false]{trace} model prediction, 
uniform (or log-uniform) distributions with reasonable range of variation are deemed sufficient for the model parameters.
Hence, the \emph{prior uncertainty propagation} step constitutes of generating samples for each model parameter from its prior \gls[hyper=false]{pdf}, 
carrying out a set of \gls[hyper=false]{trace} simulations of the \gls[hyper=false]{feba} model using the sampled parameters, 
and analyzing the results to quantify the uncertainty of the model prediction.

For a given sampled perturbation factor, one of three modes of perturbation is possible: \emph{additive, multiplicative,} and \emph{substitutive}.
In the additive mode, the sampled perturbation factor is added to the nominal parameter value of the \gls[hyper=false]{trace} model.
In the multiplicative mode, the sampled perturbation factor is multiplied by the nominal parameter value.
Finally, in the substitutive mode, the sampled perturbation factor directly substitutes for the nominal parameter value.

A tool to assist the uncertainty propagation of model parameters in a \gls[hyper=false]{trace} model is developed in the Python programming language.
This tool, detailed in Appendix~\ref{app:trace_simexp}.

As show in Table~\ref{tab:trace_model_parameter_1} and Table~\ref{tab:trace_model_parameter_2}, the list of perturbation factors involves the perturbation of various model parameters related to:
\begin{enumerate}
	\item \gls[hyper=false]{th} boundary conditions
  \item material properties
\end{enumerate}
\begin{sidewaystable}

\caption{Selected \glsentryshort{trace} model parameter perturbation factors and their range of variations, continued in Table~\ref{tab:trace_model_parameter_2}}
\label{tab:trace_model_parameter_1}

\centering
\newcolumntype{Y}{>{\RaggedRight\arraybackslash}X}
%\begin{tabularx}{\textwidth}{@{}c|>{$}c<{$}|>{$}c<{$}|Y|Y|Y|Y|Y@{}}
\begin{tabularx}{0.985\textwidth}{@{}cccc>{$}c<{$}>{$}c<{$}c@{}}
\toprule
\tableheadline{No.} & \tableheadline{Parameter} & \tableheadline{Description} & \tableheadline{Distribution} & \tableheadline{Range of}  & \tableheadline{Nominal} & \tableheadline{Mode of} \\
                    & \tableheadline{ID}        &                             &                              & \tableheadline{Variation} & \tableheadline{Value}   & \tableheadline{Perturbation} \\
\midrule
1  & \texttt{break\_ptb}  	& Outlet pressure 								& Uniform 		& [0.90, 1.10]   & 1.0 					& Multiplicative \\ 
2  & \texttt{fill\_tltb}  	& Inlet water temperature $[K]$ 	& Uniform 		& [-5.00, +5.00] & 0.0 \, [K] 	& Additive \\ 
3  & \texttt{fill\_vmtbm}	 	& Inlet water velocity          	& Uniform 		& [0.90, 1.10]   & 1.0 					& Multiplicative \\ 
4  & \texttt{pwr\_rpwtbr} 	& Heater rod power             	 	& Uniform 		& [0.90, 1.05]   & 1.0 					& Multiplicative \\ 
\midrule
5  & \texttt{nic\_cond} 		& Conductivity (Nichrome) 						& Uniform & [0.95, 1.05] & 1.0 	& Multiplicative \\ 
6  & \texttt{nic\_cp} 			& Specific heat (Nichrome)	 					& Uniform & [0.95, 1.05] & 1.0 	& Multiplicative \\ 
7  & \texttt{nic\_emis} 		& Emissivity (Nichrome) 							& Uniform & [0.90, 1.00] & 0.95	& Substitutive \\ 
8  & \texttt{mgo\_cond} 		& Conductivity (MgO)									& Uniform & [0.80, 1.20] & 1.0 	& Multiplicative \\
9  & \texttt{mgo\_cp} 			& Specific heat (cp) 									& Uniform & [0.80, 1.20] & 1.0 	& Multiplicative \\ 
10 & \texttt{vessel\_epsw} 	& Wall roughness $[m]$					  		& Uniform & [6.10\times 10^{-7}, 2.44\times 10^{-6}] & 1.5 \times 10^{-6} \, [m] & Multiplicative \\ 
11 & \texttt{ss\_cond} 			& Conductivity (stainless steel)			& Uniform & [0.95, 1.05] & 1.0 	& Multiplicative \\ 
12 & \texttt{ss\_cp} 				& Specific heat (stainless steel)			& Uniform	& [0.95, 1.05] & 1.0 	& Multiplicative \\ 
13 & \texttt{ss\_emis} 			& Emissivity (stainless steel)				& Uniform & [0.56, 0.94] & 0.84 & Substitutive \\ 
\midrule
14 	& \texttt{kGridSV} 			& Spacer grid $\Delta p$ coefficient							& Uniform 		& [0.25, 1.75] & 1.0 & Multiplicative \\ 
15  & \texttt{gridHTEnh} 		& Spacer grid \gls[hyper=false]{htc} enhancement	& Log-Uniform & [0.90, 1.10] & 1.0 & Multiplicative \\ 
\bottomrule

\end{tabularx}
\end{sidewaystable}


\begin{sidewaystable}

\caption{Selected \glsentryshort{trace} model parameter perturbation factors and their range of variations, continued from Table~\ref{tab:trace_model_parameter_1}}
\label{tab:trace_model_parameter_2}

\centering
\newcolumntype{Y}{>{\RaggedRight\arraybackslash}X}
%\begin{tabularx}{\textwidth}{@{}c|>{$}c<{$}|>{$}c<{$}|Y|Y|Y|Y|Y@{}}
\begin{tabularx}{0.90\textwidth}{@{}cccc>{$}c<{$}>{$}c<{$}c@{}}
\toprule
\tableheadline{No.} & \tableheadline{Parameter} & \tableheadline{Description} & \tableheadline{Distribution} & \tableheadline{Range of}  & \tableheadline{Nominal} & \tableheadline{Mode of} \\
                    & \tableheadline{ID}        &                             &                              & \tableheadline{Variation} & \tableheadline{Value}   & \tableheadline{Perturbation} \\
\midrule
16  & \texttt{iafbWallHTC}  	& Wall \gls[hyper=false]{htc} (\glsentryshort{iafb})							& Log-Uniform 		& [0.50, 2.00]   	& 1.0 				& Multiplicative \\ 
17  & \texttt{dffbWallHTC}  	& Wall \gls[hyper=false]{htc} (\glsentryshort{dffb})  						& Log-Uniform 		& [0.50, 2.00] 	  & 0.0 				& Multiplicative \\ 
18  & \texttt{iafbLIHTC}	 		& Liquid-interface \gls[hyper=false]{htc} (\glsentryshort{iafb})  & Log-Uniform 		& [0.25, 4.00]   	& 1.0 				& Multiplicative \\ 
19  & \texttt{iafbVIHTC} 			& Vapor-interface \gls[hyper=false]{htc}  (\glsentryshort{iafb})  & Log-Uniform 		& [0.25, 4.00]   	& 1.0 				& Multiplicative \\ 
20  & \texttt{dffbLIHTC} 			& Liquid-interface \gls[hyper=false]{htc} (\glsentryshort{dffb}) 	& Log-Uniform 		& [0.25, 4.00] 	 	& 1.0 				& Multiplicative \\ 
21  & \texttt{dffbVIHTC} 			& Vapor-interface \gls[hyper=false]{htc} (\glsentryshort{dffb})	 	& Log-Uniform 		& [0.25, 4.00] 	 	& 1.0 				& Multiplicative \\ 
22  & \texttt{iafbIntDrag} 		& Interfacial drag (\glsentryshort{iafb}) 												& Log-Uniform 		& [0.25, 4.00] 	 	& 1.0 				& Multiplicative \\ 
23  & \texttt{dffbIntDrag} 		& Interfacial drag (\glsentryshort{dffb})													& Log-Uniform 		& [0.25, 4.00]		& 1.0 				& Multiplicative \\
24  & \texttt{iafbWallDrag} 	& Wall drag (\glsentryshort{iafb}) 																& Log-Uniform 		& [0.50, 2.00] 		& 1.0 				& Multiplicative \\ 
25	& \texttt{dffbWallDrag} 	& Wall drag (\glsentryshort{dffb})			  												& Log-Uniform 		& [0.50, 2.00] 		& 1.0 				& Multiplicative \\ 
26 	& \texttt{tempQuench} 		& Quenching temperature $[K]$																			& Uniform 				& [-50.0, +50.0] 	& 0.0 \, [K]	& Additive \\ 
27 	& \texttt{transHTCWallSV}	& Wall \gls[hyper=false]{htc} (Transition boiling)								& Log-uniform			& [0.25, 4.00] 		& 1.0 				& Multiplicative \\ 
\bottomrule

\end{tabularx}
\end{sidewaystable}
