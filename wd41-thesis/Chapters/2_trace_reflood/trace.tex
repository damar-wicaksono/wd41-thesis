\section{Thermal-Hydraulics System Code \glsentryshort{trace}}\label{sec:reflood_trace}

\gls{trace} is the best-estimate system \gls{th} code developed by the \gls{usnrc} 
as a tool for light water reactor transient analysis during normal and accident scenarios.
Its development is an on-going effort 
to modernize into a single software package all previous \gls{usnrc} \gls{th} codes
that were developed separately for specific reactor types and/or applications.
This ultimately would make the code more versatile for end users and more efficient to maintain for the developer.

The hydraulic model of \gls{trace} is based on a two-fluid six-equation model, 
solving the conservation equations of mass, momentum, and energy for the liquid and vapor phases in the coolant.
These equations are described in the following.

\paragraph{Mass balance equations, liquid and gas phases:}
\begin{equation}
	\frac{\partial [(1-\alpha)\rho_l]}{\partial t} + \nabla \cdot [(1-\alpha) \rho_l \mathbf{v_l}] = - \Gamma
\label{eq:mass_balance_liquid}
\end{equation}
\begin{equation}
	\frac{\partial [\alpha \rho_g]}{\partial t} + \nabla \cdot [\alpha \rho_g \mathbf{v_g}] = \Gamma
\label{eq:mass_balance_gas}
\end{equation}
where the subscripts indicate the phase, $l$ for the liquid phase and $g$ for the gas phase (vapor); 
$\alpha$ is the void fraction; 
$\rho$ is the mass density of the respective phase;
$\mathbf{v}$ is the velocity of the respective phase.

The first term on the left hand side is the volumetric rate of change of the mass of the corresponding phase.
The second term is the the volumetric mass convection of the corresponding phase.
The term $\Gamma$ on the right hand side is the volumetric interfacial mass-transfer rate,
with a convention that it is positive for the transfer from liquid phase to gas phase.

\paragraph{Momentum balance equations, liquid and gas phases:}
\begin{equation}
	\begin{split}
		& \frac{\partial [(1-\alpha)\rho_l \mathbf{v}_l]}{\partial t} + \nabla \cdot [(1-\alpha) \rho_l \mathbf{v_l} \mathbf{v_l}] + (1 - \alpha) \nabla P = \\
		& \quad
	\end{split}
\label{eq:momentum_balance_liquid}
\end{equation}
\begin{equation}
	\begin{split}
		& \frac{\partial [\alpha \rho_g \mathbf{v}_g]}{\partial t} + \nabla \cdot [\alpha \rho_g \mathbf{v_g} \mathbf{v_g}] + \alpha \nabla P = \\
		& \quad
	\end{split}
\label{eq:momentum_balance_gas}
\end{equation}
where

\paragraph{Energy balance equations, liquid and gas phases:}
\begin{equation}
	\begin{split}
		& \frac{\partial [(1-\alpha)\rho_l(e_l + |\mathbf{v_l}|^2/2]}{\partial t} + \nabla \cdot \left[(1-\alpha) \rho_l \left(e_l+\frac{P}{\rho_l}+\frac{|\mathbf{v_l}|^2}{2}\right)\right] = \\
		&	\qquad q_{il}
	\end{split}
\label{eq:energy_balance_liquid}
\end{equation}
\begin{equation}
	\begin{split}
		 & \frac{\partial [\alpha \rho_g (e_g + |\mathbf{v_g}|^2/2]}{\partial t} + \nabla \cdot \left[\alpha \rho_g \left(e_g+\frac{P}{\rho_g}+\frac{|\mathbf{v_g}|^2}{2}\right)\right] = \\
		 & \qquad q_{ig}
	\end{split}
\label{eq:energy_balance_gas}
\end{equation}
where