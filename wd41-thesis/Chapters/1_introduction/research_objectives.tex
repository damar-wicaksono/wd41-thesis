%*********************************************************************************
\section{Objectives and Scope of the Thesis}\label{sec:intro_objectives_and_scope}
%*********************************************************************************

%--------------------------------------------------------------------------
\subsection{Statement of the Problem}\label{sub:intro_statement_of_problem}
%--------------------------------------------------------------------------

% Introductory Paragraph
The development of closure laws for reflooding described in \cite{Nelson1992,USNRC2012} showed the difficulties and the amount of assumptions used.
In a nutshell, system code development is an effort to consolidate correlations and mechanistic models, to create a phenomenological-based simulator code that can provide best-estimate results.
This consolidated effort results in a code that can simulate wide range of transients foreseen in nuclear power plant operation in a best-estimate manner.
Alas, to come up with a consistent set of closure laws is a great challenge for code developers.

% Closure Laws Difficulty, Conceptual
The closure laws required to close the two-fluid model pose particularly difficult challenges \cite{Wulff2007}.
For instance, to have a correlation of heat transfer between the wall and the fluid, temperature data from each of the constituents are needed (i.e., the wall, the liquid phase, and the gas phase).
But measuring temperature of the individual phases in an arbitrary interfacial topology has its own technical difficulties to the extend that no such data exists or available to be implemented in the closure laws.
Additionally, the experiments to obtain hydrodynamic closure laws (e.g., interfacial friction factor, wall friction factor, etc.) were generally carried out in adiabatic conditions.
As a result, this excludes the coupling of any heat transfer phenomena between the phases and the wall in such correlation.

% Closure Laws Difficult, Practical
Furthermore, during the development of a simulation code, programming considerations also came into the picture.
For robustness, simplification is often required and continuity is enforced.
Transitionary flow regime between two known (observed) flow regimes for which experimental data is not available is modeled to be the average of the two bounding regimes.
Different code development, which used different assumptions and experimental database, comes up with different set of closure laws with their own parametrization (see for instance \cite{Nelson1992} for TRAC code and \cite{Bestion1990} for CATHARE code).
Several authors have expressed their concerns about the uncertainty stemming from the closure laws \cite{Wulff2007,Petruzzi2008a,DAuria2012}.

% an Illustration
As an example of the point given above, consider that in the \gls[hyper=false]{trace} code, after some derivations the interfacial drag coefficient closure law in the inverted slug flow regime $C_{i,\text{IS}}$ is given by,
\begin{equation*}
	C_{i,\text{IS}} = \hat{x}_{m,\text{SET}} \times \frac{1}{24} \frac{\rho_g}{\text{La}} \frac{(1-\alpha)}{\alpha^{1.8}} \,\,\,;\,\,\, \hat{x}_{m,\text{SET}} = 0.75 
\label{eq:intf_drag_isf}
\end{equation*}
where $\rho_g$ is the density of the gas phase;
$\text{La}$ is the Laplace number;
$\alpha$ is the void fraction;
and $\hat{x}_{m,\text{SET}}$ is a fitting parameter.

There are several remarks about the closure law given above.
First, the second term in the right-hand side was derived from experimental data but not directly.
In the inverted slug regime, saturated liquid core breaks up into ligaments.
These ligaments are \emph{assumed} to take form as prolate ellipsoid.
The drag coefficient of distorted droplet experimental database is then \emph{assumed}.
Then to take into account the multi-particle effect, the coefficient is divided by the void fraction $\alpha$ raised to the power of $1.8$ (this, in turn, was taken from experimental data of inertial regime).
Lastly, the first term of the equation, $\hat{x}_{m,\text{SET}} = 0.75$ was put \emph{to match}, \emph{to calibrate against} the experimental data from the FLECHT-SEASET reflood facility.
This first term, although clearly \emph{non-physical}, is an important tuning parameter of the model nevertheless.
Its uncertainty should be considered in uncertainty analysis, especially when reflood is expected to occur.
Yet, no statement regarding the associated uncertainty is given.
Several other such terms exist \cite{USNRC2012}. 


% Statement of Problem
As illustrated above, it is clear that models in thermal-hydraulics system code, to a certain extent, flawed.
Various experimental programs were carried out to gain better understanding of important phenomena,
and to validate (and, as noted above, to calibrate) the models.
Series of the experiments, carried out in \glspl[hyper=false]{setf} were aimed to reproduce limited part of the transient in a selected component following a postulated scenario.
For example, in the case of reflooding, several facilities existed and data were available (FEBA, PERICLES, etc.).
But, there has not been an orchestrated effort to incorporate the accumulated data into the calibration process of the physical models, in a systematic way, while acknowledging multiple sources of the uncertainty in the process.

%--------------------------------------------------
\subsection{Objectives}\label{sub:intro_objectives}
%--------------------------------------------------

% Introductory (Overall Objective)
The overall objective of the present research is to quantify the uncertainty of physical model parameters implemented in system code in the form of closure laws.
These models are parametrized by either physical parameters or tuning parameters.
Usual practice by the code developer is to calibrate them against some experimental data without any statement of their uncertainties.
To conform with the framework of statistical uncertainty propagation, uncertainty associated with these parameters are required to be represent in the form of probability distribution.
The derivation of these distributions will be based on probabilistic modeling and available experimental data.
In accordance to this overall objective, the present thesis is divided into four sequential phases.

% Aim 1 (Global sensitivity analysis)
First is to investigate whether the current physical model in system code can be identified with the available experimental data from test facilities.
\glsfirst[hyper=false]{gsa} methodology can be used to assist in identifying which parameters can be calibrated using the available data.
A test facility might have multiple types of data and although the information content might not be the same for the different types, it might be worthwhile to consider each one of them.

% Aim 2 (Statistical Metamodeling)

% Aim 3 (Bayesian Calibration)
Third is to calibrate the physical model parameters against varous relevant experimental data.
The word \emph{to calibrate} carries a disparaging interpretation related to \emph{to tweak}.
However, using a Bayesian framework, the aim of calibration is extended to preserve the uncertainty of the parameter estimation.
Furthermore, in this framework, various sources of uncertainty can be modeled using probability distributions, including the model bias term.
At the end, the parameters of interest will be either in the form of distributions conditioned on the data or samples generated from such distributions.
They are the main ingredients in the aforementioned statistical uncertainty analysis (i.e., propagation).

% Aim 4 (Extrapolation)


%----------------------------------------
\subsection{Scope}\label{sub:intro_scope}
%----------------------------------------

% Introductory paragraph
Although the steps taken in this research can be applicable to any system code physical model calibration and validation, the attention will be focused on the models of particular importance during reflooding.,
the so-called \gls[hyper=false]{postchf} flow regimes.
The reasons for this emphasis are:

% Reason 1

% Reason 2

% Reason 3

% Reason 4

% Closing
Finally, the thermal-hydraulics system code considered in this thesis is the \glsfirst[hyper=false]{trace} code developed by the \glsfirst[hyper=false]{usnrc}.
The main reason to consider solely this particular code is the fact that \gls[hyper=false]{trace} is the thermal-hydraulics system code used for the purpose of Swiss nuclear power plant safety analysis conducted withing the \glsfirst[hyper=false]{stars} program \cite{PSI2017} at the \glsfirst[hyper=false]{psi}.