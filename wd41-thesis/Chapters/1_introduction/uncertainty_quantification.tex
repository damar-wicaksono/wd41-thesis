%*************************************************************************************************************************
\section{Uncertainty Quantification in Nuclear Engineering Thermal-Hydraulics}\label{sec:intro_uncertainty_quantification}
%*************************************************************************************************************************

% Complex Simulator
Granted, a simplicistic model cannot be expected to imitate all the important features of a complex physical phenomenon.
And yet, there is a tendency of developing and applying overly complex model, with numerous parameters and multiple non-linear relationships, for computer simulation.
This tendency has invited many critics over the year (citation needed).
In nuclear science and engineering, for instance, Zuber \cite{Zuber2001} and Wullf \cite{Wulff2007} have long critized the development and the use of multi-fluid model for thermal-hydraulics simulation as high in complexity and in maintenance cost, but low in fidelity and its usefulness.

The goal of computer simulation, complex or otherwise, is to provide prediction.
The main characteristic (and source of criticism) of using multi-parameter complex model for simulation is that the the relationship between numerous inputs and outputs becomes increasingly opaque.
The impact of changing one parameter alone, or especially together, on the prediction is hard to disentangle or intuit.
Furthermore, as appropriate values of inputs might not be fully known, they are often given simply over range of interest containing different possible permissible values.
It is thus seldom the case that one single simulation is sufficient to provide a reliable answer according to hardly any objective of computer simulation.
As analytical solution

\gls[hyper=false]{trace}

% Introductory Paragraph
In making a connection with the notion \emph{simulator}, 

%----------------------------------------------------------------------------------------------
\subsection{Statistical Uncertainty Analysis}\label{sub:intro_statistical_uncertainty_analysis}
%----------------------------------------------------------------------------------------------

% Best-estimate, limitation
As explained, best-estimate analysis uses more realistic modeling assumptions for analyzing transient behavior of \gls[hyper=false]{npp}.
It attempts as realistically as possible to describe the behavior of the relevant physical processes occur during the plant transient.
And yet, even the best available understanding of the physical process is still limited.
Understanding of complex phenomena might not yet adequate and data support for some processes can be very limited.
Simplifying assumptions, approximations, and expert judgments to some degree are unavoidable and still required to have a complete analysis.

% Best-estimate, plus uncertainty

% Source of possible uncertainties

% Statistical uncertainty analysis, Inputs as random variables 

% Source of uncertainty, initial and boundary condition

% Source of uncertainty, physical model parameters

% Inverse uncertainty

% Connection to PREMIUM Benchmark

%---------------------------------------------------------------
\subsection{OECD/NEA PREMIUM Benchmark}\label{sub:intro_premium}
%---------------------------------------------------------------

% Introductory paragraph
The \gls[hyper=false]{premium} benchmark was an activity launched by the \gls[hyper=false]{oecd}/\gls[hyper=false]{nea} in $2012$ and concluded in $2016$ with the aim to advance the methods for quantifying the uncertainties associated with the physical model parameters in \gls[hyper=false]{th} system codes.
It was the continuation of the previous project \gls[hyper=false]{bemuse}, which concetrated on the propagation and sensitivity analysis of the input uncertainties in large scale simulation (large break \gls[hyper=false]{loca}).
The main finding of \gls[hyper=false]{bemuse} can be found in \cite{Perez2011}.
The emphasis of the \gls[hyper=false]{premium} benchmark was placed on the derivation of the model parameters uncertainty and their validation.

% Scope of the Project

% Main Findings

%-------------------------------------------------------
\subsection{GRS Methodology}\label{sub:intro_grs_method}
%-------------------------------------------------------

%-----------------------------------------------------------
\subsection{FFTBM Methodology}\label{sub:intro_fftbm_method}
%-----------------------------------------------------------

%-------------------------------------------------------
\subsection{CIRC\'E}\label{sub:intro_circe_method}
%-------------------------------------------------------
