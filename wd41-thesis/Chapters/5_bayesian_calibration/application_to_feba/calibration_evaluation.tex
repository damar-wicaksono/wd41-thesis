%***********************************************************************
\subsection{Calibration Evaluation}\label{sub:bc_calibration_evaluation}
%***********************************************************************

% FEBA Test No. 216 Posterior w/ bias term Uncertainty Propagation, TC
\clearpage
\begin{sidewaysfigure}
	\centering
	\includegraphics[width=0.90\textwidth]{../figures/chapter5/figures/plotTraceUQPosteriorAllDiscCenteredTC216}
		\captionof{figure}[Prior uncertainty propagation of FEBA Test No. $216$ for the pressure drop output ($TC$).]{Uncertainty propagation of the parameters uncertainty of \gls[hyper=false]{feba} Test No. $216$ for the cladding temperature output ($TC$) at different axial locations using \gls[hyper=false]{trace}. The uncertainty bounds refer to the symmetric ($95\%$) probability: dark gray, gray, and light gray correspond to the prior, correlated posterior, and independent posterior samples of the parameters, respectively. Solid lines, dashed lines, and crosses indicate the simulation with the nominal parameters values, the median of the posterior uncertainties, and the experimental data, respectively.}
	\label{fig:ch5_plot_trace_uq_posterior_alldisccentered_tc_216}
\end{sidewaysfigure}
\clearpage

% FEBA Test No. 216 Posterior w/ bias term Uncertainty Propagation, DP
\bigfigure[pos=tbhp,
           opt={width=1.0\textwidth},
           label={fig:ch5_plot_trace_uq_posterior_alldisccentered_dp_216},
           shortcaption={Prior uncertainty propagation of FEBA Test No. $216$ for the pressure drop output ($DP$).}]
{../figures/chapter5/figures/plotTraceUQPosteriorAllDiscCenteredDP216}
{Uncertainty propagation of the parameters uncertainty of \gls[hyper=false]{feba} Test No. $216$ for the pressure drop output ($DP$) at different axial segments using \gls[hyper=false]{trace}. The uncertainty bounds refer to the symmetric ($95\%$) probability: dark gray, gray, and light gray correspond to the prior, correlated posterior, and independent posterior samples of the parameters, respectively. Solid lines, dashed lines, and crosses indicate the simulation with the nominal parameters values, the median of the posterior uncertainties, and the experimental data, respectively.}

% FEBA Test No. 216 Posterior w/ bias term Uncertainty Propagation, CO
\begin{figure}[!bth]
    \centering
    \includegraphics[width=0.5\textwidth]{../figures/chapter5/figures/plotTraceUQPosteriorAllDiscCenteredCO216}
    \caption[Prior uncertainty propagation of FEBA Test No. $216$ for the pressure drop output ($CO$).]{Uncertainty propagation of the parameters uncertainty of \gls[hyper=false]{feba} Test No. $216$ for the liquid carryover output ($TC$) using \gls[hyper=false]{trace}. The uncertainty bounds refer to the symmetric ($95\%$) probability: dark gray, gray, and light gray correspond to the prior, correlated posterior, and independent posterior samples of the parameters, respectively. Solid lines, dashed lines, and crosses indicate the simulation with the nominal parameters values, the median of the posterior uncertainties, and the experimental data, respectively.}
    \label{fig:ch5_plot_trace_uq_posterior_alldisccentered_co_216}
\end{figure}

% Calibration Score vs Informativeness
\clearpage
\begin{sidewaysfigure}
	\centering
	\includegraphics[width=0.95\textwidth]{../figures/chapter5/figures/plotCalibInfo}
		\captionof{figure}[Calibration score vs. Informativeness for different posterior samples propagated on all the FEBA tests.]{Calibration score vs. Informativeness for different posterior samples propagated on all the \gls[hyper=false]{feba} tests. Vertical lines indicate the informativeness of the prior uncertainty (defined as $0$) while the horizontal lines indicate the initial Calibration score (i.e., that of the prior).}
	\label{fig:ch2_plot_trace_uq_prior_tc_218}
\end{sidewaysfigure}
\clearpage


