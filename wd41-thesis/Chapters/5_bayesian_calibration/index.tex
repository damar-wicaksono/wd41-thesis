%*************************************************************************************************************************************************
\chapter[Bayesian Calibration]{Bayesian Calibration of Computer Model: Bridging Model \& Data under Uncertainty}\label{ch:bayesian_calibration}
%*************************************************************************************************************************************************

% Link paragraph
In Chapter~\ref{ch:gsa}, a sensitivity analysis method was employed to better understand the inputs/outputs relationship in a computer simulation model with uncertain inputs.
The method was also able to reduce the size of the problem by screening out non-influential inputs.
Chapter~\ref{ch:gp_metamodel} then developed a fast approximation to evaluate the output at any given input point, in anticipation of the high cost of the calibration approach presented in this chapter.
The respective methods were exemplified by their application to a \gls[hyper=false]{trace} reflood simulation model whose inputs were uncertain, as assumed in Chapter~\ref{ch:trace_reflood}.

% Focus paragraph
This chapter deals with a statistical framework for calibrating the inputs of a simulation model.
The framework casts the calibration problem as a statistical inverse problem where the initial (prior) uncertainties of the inputs are updated based on available observed data.
It considers the a priori uncertainties in the inputs and in the experimental data, as well as the possible bias of the model.
Acknowledging them,
the inputs uncertainties are then coherently updated via the Bayes' theorem resulting in an updated (posterior) probability density.
The updated uncertainty of the inputs can then be propagated through the simulation model to quantify the prediction uncertainty.

% Overview paragraphs
Section~\ref{sec:bc_statistical_framework} first presents the statistical framework for the problem of computer model calibration, 
while Section~\ref{sec:bc_modular} elaborates further the formulation of the calibration problem through a modularized approach.
Each module represents a model for the data-generating processes involved: the computer simulation model, the experimental data, and the model discrepancy.
Those are the ingredients in the formulation of the posterior probability density.
The posterior density is often a complex highly multi-dimensional function, which makes it difficult to work with.
Section~\ref{sec:bc_mcmc} presents a simulation method (i.e., \glsfirst[hyper=false]{mcmc} simulation) to directly generates representative samples from the posterior density.
These samples can be used to approximate the posterior density or for uncertainty propagation.
Important aspects of analyzing samples of a Markov chain are presented in Section~\ref{sec:bc_mcmc_diagnostic}.
Section~\ref{sec:bc_application_to_feba} then discusses the application of the approach to the \gls[hyper=false]{feba} \gls[hyper=false]{trace} reflood simulation model to constrain the prior uncertainty range of the model parameters based on the available experimental data.
To do so, different types of experimental data are used and their ability to constrain the prior range is investigated.
Section~\ref{sec:bc_chapter_summary} concludes the chapter.

\section{Statistical Framework}\label{sec:gp_statistical_framework}
%**************************************************************************
\section{Bayesian Formulation of Calibration Problem}\label{sec:bc_modular}
%**************************************************************************

% Introductory paragraph
The Bayesian framework for model calibration begins by constructing a probabilistic model of $y_E$ given in an additive formulation of Eq.~(\ref{eq:bc_observation_true}). 
That is, it aims at formulating the data generating process $\mathcal{Y}_E(\bm{x}_c; \bm{\lambda})$.
This model implies that the experimental data $y_E$ taken at particular $\bm{x}_c$ observed at $\bm{\lambda}$ is a realization of a stochastic process.
Furthermore, this probabilistic modeling entails casting any \emph{uncertain} element in Eq.~(\ref{eq:bc_observation_true}) either as random variable or stochastic process.

%-----------------------------------------------------------------------------
\subsection{Probabilistic Model for the Model Bias}\label{sub:bc_modular_bias}
%-----------------------------------------------------------------------------

% Introductory Paragraph
Recall the relationship between the true system response and its prediction by a simulator (Eq.~(\ref{eq:bc_true_simulation})) rearranged below:
\begin{equation*}
    \delta (\bm{x}_c, \boldsymbol{\lambda}) = y_T(\bm{x}_c, \boldsymbol{\lambda}) - y_M (\bm{x}_c, \hat{\bm{x}}_m, \boldsymbol{\lambda})
\end{equation*}
where the prediction $y_M$ is made using the best but unknown value of the model parameters.
As such, the model bias function $\delta$ represents a possible systematic difference between the true system response and the simulator prediction that still remains, even from using a simulator with the \emph{best} set of model parameter values.
\marginpar{Model bias, possible origins}
Possible sources for this bias are missing physics in the constituent physical models, numerical approximations, or any other simplifications presence in the simulator whose effects on the prediction are unknown a priori.
As such, the bias term tends to be systematic and dependent on the controllable input $\bm{x}_c$ and the observation layout $\boldsymbol{\Lambda}$ \cite{Reichert2012}.
Note that, strictly speaking, there is a dependence of $\hat{\bm{x}}_m$ on $\delta$, but this dependence is suppress from the notation; $\hat{\bm{x}}_m$, though unknown, should in principle be a unique set of values valid for all $\bm{x}_c$ \cite{Bayarri2007,Arendt2012}.

% Probability Model for Model Discrepancy, Gaussian Process for Model Discrepancy
The unknown model bias function $\delta$ can be represented as a random function $\mathcal{D} (\circ)$,
\begin{equation}
        (\mathcal{Y}_T - y_M(\bm{x}_c, \hat{\bm{x}}_m, \bm{\lambda})) \equiv \mathcal{D}(\bm{x}_c, \bm{\lambda}) \thicksim p(\delta | \bm{\psi}_{\delta}, \bm{x}_c, \bm{\lambda})
\label{eq:bc_data_generating_bias}
\end{equation}
where $\bm{\psi}_{\delta}$ is the parametrization of the probability density describing the bias at $\bm{x}_c$ and $\bm{\lambda}$.

Casting the unknown model bias term as a stochastic process is the salient feature of Bayesian calibration framework proposed by Kennedy and O'Hagan \cite{Kennedy2001}.
\marginpar{Gaussian process formulation}
In particular, a stationary \glsfirst[hyper=false]{gp} $\mathcal{D} (\bm{x}_c, \bm{\lambda})$ on $\mathbf{X}_C \subseteq \mathbb{R}^{D_c}$ and on $\bm{\Lambda}$ is used to represent the term.
That is,
\begin{equation}
        \mathcal{D}(\circ, \circ) \thicksim \mathcal{GP}(m_\delta(\circ,\circ; \bm{\psi}_{\delta}), K_\delta((\circ,\circ), (\circ, \circ); \bm{\psi}_{\delta}))
\label{eq:bc_data_generating_bias_gp}
\end{equation}
where $m_\delta$ and $K_\delta$ are the mean function and the covariance function of the \gls[hyper=false]{gp}, respectively;
and $\bm{\psi}_{m}$ is the hyper-parameters associated with the specification of the \gls[hyper=false]{gp} for the model bias function (e.g., its covariance kernel, see Chapter~\ref{ch:gp_metamodel}).
Under a \gls[hyper=false]{gp} formulation, the notion of \emph{systematic} bias mentioned previously is described statistically in terms of the mean and the covariance \cite{Reichert2012}.

For a selected values of $\bm{x}_c$ and $\bm{\lambda}$, the \gls[hyper=false]{gp} becomes a Gaussian random variable,
\begin{equation}
        \mathcal{D}(\bm{x}_c, \bm{\lambda}) \thicksim \mathcal{N}(m_\delta(\bm{x}_c, \bm{\lambda}; \bm{\psi}_{\delta}), s^2_\delta(\bm{x}_c, \bm{\lambda}; \bm{\psi}_{\delta}))
\label{eq:bc_data_generating_bias_gp_restricted}
\end{equation}
where $s^2_\delta$ is the standard deviation at controllable input $\bm{x}_c$ observed at $\bm{\lambda}$, under the parametrization $\bm{\Psi}_\delta$ of the \gls[hyper=false]{gp}.
Finally, for observations on multiple combinations of the controllable inputs and/or the complete observation layout $\bm{\Lambda}$, the \gls[hyper=false]{gp} becomes a multivariate Gaussian random variable, taking into account correlations of the bias at different elements of the observation layout,  
\begin{equation}
        \mathcal{D}(\bm{x}_c, \bm{\Lambda}) \thicksim \mathcal{N}(m_\delta(\bm{x}_c, \bm{\Lambda}; \bm{\psi}_{\delta}), \Sigma_\delta(\bm{x}_c, \bm{\Lambda}; \bm{\psi}_{\delta}))
\label{eq:bc_data_generating_bias_gp_multivariate}
\end{equation}
where $\Sigma_\delta$ is the (symmetric) covariance matrix of the bias at controllable input $\bm{x}_c$ observed on $\bm{\Lambda}$, under the parametrization $\bm{\Psi}_\delta$ of the \gls[hyper=false]{gp}.
The size of the matrix is $P \times P$, with $P$ the sum of the number of different combinations of the controllable inputs and the number of elements in the observation layout.

% Why bias, unbiased model
Incorporating bias term in the calibration procedure is important to avoid overfitting in the model parameters estimates.
\marginpar{Model without bias, illustrated}
To illustrate this idea, consider a calibration of a computer simulator without the presence of bias, with a single uncertain model parameter and a single controllable input $x_c$ as shown in Fig.~\ref{fig:ch5_plot_illustrate_bias_1}.
The thin black lines between the two bounding thick black lines indicate simulator prediction at different values of the model parameter.
As can be seen, the range of the model parameter values can in principle be constrained to match the observed data (crosses) within the observation uncertainty.
Furthermore, the range of the model parameters will increasingly become smaller with increasing number of data (such that the associated observation uncertainty becomes increasingly narrow as well).
In other words, the calibrated model parameter converges to the ``true'' value \cite{Bayarri2007,OHagan2013,Brynjarsdottir2014}.
This parameter value will be valid for prediction outside the calibration domain (i.e., extrapolation at different values of controllable inputs where no data has been observed).
\normdoublefigure[pos=tbhp,
                  mainlabel={fig:ch5_plot_illustrate_bias},
                  maincaption={Illustration of predictions made by computer simulator with and without bias, both with an uncertain model parameter and a controllable input $x_c$. Crosses are the observed data along with the associated uncertainty taken at different controllable inputs $x_c$. Bold lines are the simulator prediction using the maximum and minimum of the uncertain model parameter, thin lines are the prediction with varying values of the model parameter, and dotted lines are the prediction outside the calibration domain.},%
									mainshortcaption={Illustration of predictions made by computer simulator with and without bias, both with a single uncertain model parameter and a single controllable input $x_c$.},
                  leftopt={width=0.45\textwidth},
                  leftlabel={fig:ch5_plot_illustrate_bias_1},
                  leftcaption={Model without bias and with a single uncertain model parameter.},
                  rightopt={width=0.45\textwidth},
                  rightlabel={fig:ch5_plot_illustrate_bias_2},
                  rightcaption={Model with bias and with a single uncertain model parameter.}
                 ]
{../figures/chapter5/figures/plotIllustrateBias_1.pdf}
{../figures/chapter5/figures/plotIllustrateBias_2.pdf}

% Why bias, biased model
On the other hand,
some simulators would have apparent bias such that their predictions would remain inconsistent with the observed data, regardless the choice of the model parameter value (Fig.~\ref{fig:ch5_plot_illustrate_bias_2}).
\marginpar{Model with bias, illustrated}
Calibration can still be conducted such that the discrepancy between data and prediction is minimized in some sense (i.e., some kind of \emph{best-fitting} model parameter value).
The calibrated parameter would be able to predict calibration data well, but not for prediction outside the calibration domain.
The situation becomes more problematic when more precise data becomes available such that uncertainty associated with the observed data becomes narrower.
In that situation, the uncertainty associated with the calibrated model parameter will also become narrower up to a point value.
\marginpar{Overfitting}
This illustrates the two symptoms of overfitting the model parameter: the calibrated model parameter is \emph{biased} (i.e., having a wrong value) and the distribution characterizing its uncertainty is \emph{degenerate} (i.e., increasingly sure on the wrong value with higher precision of the observed data).
The latter symptom is particularly troublesome as it inflates the degree of confidence one has on the prediction.

% Physics-based Simulator
The situation of a biased model is prevalent in complex physics-based simulators, whose constituent physical models were developed using scientific theory and supported by experimental data.
\marginpar{Physics-based simulators}
This approach forms the scientific basis for making prediction, especially in the region outside the calibration domain \cite{Arhonditsis2008}.
It is hoped that such an approach would be more robust than using purely statistical model of observed data \cite{Bayarri2007,Reichert2012}.
However, certain degree of simplifications from numerical approximation to ignored physical process due to lack of knowledge are expected to persist.
Furthermore, the strong scientific foundation and the experimental data support of physical models often only apply to the separate constituent models of a complex simulator \cite{Campbell2006}.
In practice, the simulator consolidates numerous models to simulate the behavior of a (more) complex system outside the calibration domain.
As such, it can also be expected that the predictions from such simulators would exhibit certain degree of bias (from the true value) that is unknown a priori.

One might argue that if a model is known to be biased it simply requires more developmental effort to correct the bias by putting additional models for the missing physical processes.
However, as argued in \cite{Arhonditsis2008,Wulff2007}, this approach might not be the best solution as additional models often require even more model parameters to be calibrated and thus call for even more supporting data that cannot be met.
Additionally, as noted in \cite{Campbell2006,Bayarri2007,Brynjarsdottir2014}, it is often impractical (or unrealistic) for an analyst to revise the inner workings of a large complex simulator.
Yet, to wait until a better simulator is available before making any prediction is simply not constructive.

% Why model bias might help
In a Bayesian framework, the statistical description of the model bias term can potentially alleviate the problem overfitting.
\marginpar{Statistical description of the model bias}
Because the model parameters and the model bias are not fully identifiable according to Eq.~(\ref{eq:bc_true_simulation})\footnote{that is, without further prior information, arbitrary choice of $\hat{\bm{x}}_m$ fits the data perfectly well for arbitrary choice of $\delta$. In other words, the two terms are \emph{confounds}.}, 
having more precise data will not make the uncertainty associated with the calibrated model parameters to collapse (i.e., its distribution becomes degenerate) \cite{Bayarri2007,Brynjarsdottir2014}.
Whether the calibrated model parameters and the associated uncertainty are applicable for extrapolation outside the calibration domain, however, depends on whether the bias term is modeled properly \cite{Bayarri2007,Arhonditsis2008,Arendt2012,OHagan2013,Brynjarsdottir2014,Ling2014}.
Thus, such a statistical description of the model bias is not a magic bullet in the calibration of a biased model.
It does, however, provides additional flexibility in incorporating either prior knowledge or a prior expectation regarding model deficiency.
%This way, it also provides additional channel for assessing the suitability of the simulator to make prediction.

% Physical Parameter vs. Tuning Parameter, some philosophical issue
At this point, it is worth revisiting the meaning of calibrated parameters in a simulator with bias.
In a simulator without bias it is straightforward to justify the calibrated model parameters as the ``true'' parameter values of the specified model.
\marginpar{physical parameters, ``true'' values}
If the model is physics-based then the parameters also correspond to a physical parameter.
As argued in \cite{OHagan2013,Brynjarsdottir2014}, physical parameters often have meaning outside the world described by the model where the parameters currently reside.
Furthermore, having a true value, such physical parameters would be generally applicable to extrapolation outside the calibration domain.

On the contrary, as illustrated in one of the examples above, calibrated model parameters in a simulator with bias act as best-fitting parameters that allow the simulator to fit, in some sense, the calibration data.
\marginpar{Tuning parameters, best-fitting values}
Incorporating model bias term might help in alleviating the problem of overfitting, but the a priori arbitrariness of the model bias term confounds with the model parameters itself, making the resulting calibrated model parameters more difficult to interpret \cite{Higdon2004}.
As such, in practice, it is important to emphasize that calibrated model parameters in a simulator with bias will simply be optimal under particular assumptions (e.g., criteria, model bias term, etc.) \cite{Campbell2006}.
Ref.~\cite{Brynjarsdottir2014} went further by arguing that such model parameters (tuning) had limited scientific values and would not help for extrapolation.

This thesis takes a more pragmatic approach regarding this dichotomy: the distinction is rather irrelevant.
It is awkward to discuss the true and wrong values of model parameters if the model itself is considered biased (i.e., wrong).
\marginpar{A pragmatic view}
In such cases, the notion of true parameter values is difficult to justify, the model parameters might not have strict physical meaning and may not be of interest in their own right.
And yet, in a complex physics-based simulator (where possible systematic bias cannot be excluded), many of these model parameters are being used in conditions different from their calibration domain, regardless of the conceptual distinction (e.g., Refs.~\cite{Arendt2012,USNRC2012}).
Thus, the calibration of model parameters based on the available experimental data should be aimed such that the simulator remains applicable when it is applied outside its calibration domain\footnote{or more eloquently in the words of Leamer \cite{Saltelli2006}, the resulting uncertainty associated with calibrated model parameters is:``...wide enough to be credible and the corresponding interval of inferences is narrow enough to be useful''.}.
The Bayesian framework accommodates this aim of calibration in a flexible manner by taking into account multiple source of uncertainties through selection of prior uncertainties both for model parameters and for model bias which eventually results in the associated posterior uncertainties.

%------------------------------------------------------------------------------------
\subsection{Probabilistic Model for the Experimental Data}\label{sub:bc_modular_data}
%------------------------------------------------------------------------------------

% Introductory paragraph
Now recall the relationship between the true system response and its observation through a measurement (Eq.~(\ref{eq:bc_observation_true})),
\begin{equation*}
    y_E(\bm{x}_c, \boldsymbol{\lambda}) = y_T (\bm{x}_c, \boldsymbol{\lambda}) + \epsilon
\end{equation*}
The observation error term $\epsilon$ represents any possible error during the measurement process, either from the imprecision of the instrument or any other residual variability of the experiment.
\marginpar{Observation error, possible origins}
This variability, in turn, might be due to the inherently stochastic nature of the physical process (irreducible) or unrecognized and uncontrolled variables (reducible) \cite{Kennedy2001}.

% Generic Formulation
Because this term is considered unknown, a stochastic process is defined on the observation layout,
\begin{equation}
        \mathcal{E}(\bm{\lambda}) \thicksim p(\epsilon | \psi_{\epsilon}, \bm{\lambda})
\label{eq:bc_data_generating_exp}
\end{equation}
where $\bm{\psi}_{\epsilon}$ is the parametrization of the \gls[hyper=false]{pdf} describing the observation error $\bm{\lambda}$.
That is, it depends on which response is observed, as well as where and when it is observed.

% Independence assumption and its justification
An important assumption made on the distribution of the observation error is that it is independent conditional on the true value of the system response.
\marginpar{Conditional independence}
One can argue that the measurement data points taken from a spatio-temporal physical process would have (perhaps complicated) correlation structure among them.
But intuitively, as argued in \cite{Wikle2001}, this structure becomes much simplified once the true value is known;
it can mainly be attributed to the residual variability and instrument precision with a simpler description.
The true system response itself is already separately formulated in terms of the simulator prediction and a model bias term (Eq.~(\ref{eq:bc_true_simulation})).
As such, any possible complicated structure of the error (either bias or correlation) is already assigned to the model bias formulation and assuming a simpler measurement error model (i.e., independent) is sufficient \cite{Wikle2001}.
At the same time, as noted in \cite{Kennedy2001,Bayarri2007}, it will be difficult to distinguish two correlation structures separately for the model bias and observation error based on the data alone.

% Gaussian Assumption, some comments
The particular distribution of the observation error is often assumed to be a Gaussian in the
\marginpar{Gaussian observation error}
applied literature \cite{Wikle1998,Wikle2001,Kennedy2001,Bayarri2007,Arhonditsis2008},
\begin{equation}
        \mathcal{E} \thicksim \mathcal{N}(0, \sigma_{obs}^2(\bm{\lambda}))
\label{eq:bc_data_generating_exp_gaussian_1}
\end{equation}
or equivalently following the conditional independence assumption explained above,
\begin{equation}
  \mathcal{Y}_E | \mathcal{Y}_T = y_T(\bm{x}_c, \boldsymbol{\lambda}) \thicksim \mathcal{N}(y_T(\bm{x}_c, \boldsymbol{\lambda}), \sigma_{obs}^2(\bm{\lambda}))
\label{eq:bc_data_generating_exp_gaussian_2}
\end{equation}
where $\sigma_{obs}^2$ is the variance of the Gaussian distribution and the only hyper-parameter of this observation error specification.
The value of the variance depends on the element of the observation layout $\bm{\lambda}$.
Eq.~(\ref{eq:bc_data_generating_exp_gaussian_1}) implies that the observation is taken without bias and the error is independent (but need not be identically distributed) Gaussian random variable.

%---------------------------------------------------------------------------------
\subsection{Probabilistic Model for the Simulator}\label{sub:bc_modular_simulator}
%---------------------------------------------------------------------------------

% Introductory paragraph
For a deterministic simulator $y_M$,
the probabilistic modeling of the bias term $\delta$ and the observation error term $\epsilon$ are enough to formulate a probabilistic model for the experimental observation $\mathcal{Y}_E$.
However, following the development taken in Chapter~\ref{ch:gp_metamodel},
a \glsfirst[hyper=false]{gp} can also be used to represent a deterministic simulator using an explicit formulation of a stochastic process.
The prediction made by the simulator at particular values of $\bm{x}_c$, $\hat{\bm{x}}_m$, and $\bm{\lambda}$ is then given by,
\begin{equation}
	\mathcal{Y}_M (\bm{x}_c, \hat{\bm{x}}_m; \bm{\lambda}) \thicksim  \mathcal{N}(m(\bm{x}_c, \hat{\bm{x}}_m; \bm{\psi}_{m}, \bm{\lambda}), s^2(\bm{x}_c, \hat{\bm{x}}_m; \bm{\psi}_{m}, \bm{\lambda}))
\label{eq:bc_data_generating_simulator_gp}
\end{equation}
where $m$ and $s^2$ is the kriging mean and the kriging variance, respectively (see Section~\ref{sec:gp_metamodeling});
and $\bm{\psi}_{m}$ is the hyper-parameters associated with the specification of the \gls[hyper=false]{gp} (e.g., its covariance kernel).

This step is taken especially if the simulator is computationally expensive to evaluate and only a limited number of simulator runs can be afforded \cite{Kennedy2001,Bayarri2007,Arendt2012}.
The probability model in Eq.~(\ref{eq:bc_data_generating_simulator_gp}) then becomes an approximation to the actual simulator (i.e., a \gls[hyper=false]{gp} metamodel).
Furthermore, as explained in the Chapter~\ref{ch:gp_metamodel}, the uncertainty associated with a prediction by the metamodel at an arbitrary input point stems from the fact that the simulator itself was not run at that input.
This prediction is based on the outputs of which the simulator was run (i.e., the training data)\footnote{the statement conditional on the training data in Eq.~(\ref{eq:bc_data_generating_simulator_gp}), i.e., $\mathcal{Y}_M (\bm{x}_c, \hat{\bm{x}}_m; \bm{\lambda}) | \mathcal{Y}(\mathbf{DM})$ has been implicitly assumed.}.
As such, in this case, the uncertainty has an epistemic interpretation.

%-----------------------------------------------------------------------------
\subsection{Posterior of the Model Parameters}\label{sub:bc_modular_posterior}
%-----------------------------------------------------------------------------

% Introductory Paragraph
Summarizing the above discussions for a deterministic simulator $y_M$,
\marginpar{Data generating process, general}
\begin{equation}
    \begin{split}
				& \mathcal{Y}_M \equiv \mathcal{Y}_M \thicksim p(y_M | \hat{\bm{x}}_m, \bm{x}_c; \bm{\lambda}) = \delta_d (y_M - y_M(\hat{\bm{x}}_m, \bm{x}_c; \bm{\lambda})) \\
        & (\mathcal{Y}_T - \mathcal{Y}_M) \equiv \mathcal{D}(\bm{x}_c; \bm{\lambda}) \thicksim p(\delta | \bm{\psi}_{\delta}, \bm{x}_c; \bm{\lambda}) \\
        & (\mathcal{Y}_E - \mathcal{Y}_T) \equiv \mathcal{E}(\bm{\lambda}) \thicksim p(\epsilon | \bm{\psi}_{\epsilon}; \bm{\lambda}) \\
    \end{split}
\label{eq:bc_data_generating_models}
\end{equation}
where $\delta_d$ is the Dirac delta function indicating that the simulator prediction is exact (i.e., a \emph{degenerate} density).

Suppose that the form of the densities in Eq.~\ref{eq:bc_data_generating_models} are already given,
then the stochastic process $\mathcal{Y}_E$ is obtained by adding the terms on the right hand side of Eq.~\ref{eq:bc_true_simulation}.
Assuming that they are independent, the \gls[hyper=false]{pdf} of $\mathcal{Y}_E$ is defined as the convolution of the terms,
\begin{equation}
  \begin{split}
  p(y_E | & \bm{\psi}_{\delta}, \bm{\psi}_{\epsilon}, \hat{\bm{x}}_m, \bm{x}_c ; \bm{\lambda}) = \ldots \\
	& (p(y_M(\hat{\bm{x}}_m, \bm{x}_c; \bm{\lambda})) * p(\delta | \bm{\psi}_{\delta}, \bm{x}_c; \bm{\lambda}) * p(\epsilon | \bm{\psi}_{\epsilon}; \bm{\lambda}))(y_E)
  \end{split}
\label{eq:bc_additive_convolution}
\end{equation}
where $*$ is the symbol for the convolution operation.

% Normal Approximation
Following the Gaussian distribution formulations for the model bias, the observation error, and the
\marginpar{Data generating process, Gaussian}
simulator approximation, a normal likelihood for the calibration problem can be obtained as follows,
\begin{equation}
    \begin{split}
				& \mathcal{Y}_E = \mathcal{Y}_M + \mathcal{D} + \mathcal{E} \\
				& \mathcal{Y}_M (\bm{x}_c, \hat{\bm{x}}_m; \bm{\lambda}) \thicksim \mathcal{N}(m_M(\bm{x}_c, \hat{\bm{x}}_m; \bm{\psi}_{m}, \bm{\lambda}), s_M^2(\bm{x}_c, \hat{\bm{x}}_m; \bm{\psi}_{m}, \bm{\lambda})) \\
        & \mathcal{D}(\bm{x}_c; \bm{\lambda}) \thicksim \mathcal{N}(m_\delta(\bm{x}_c; \bm{\psi}_\delta, \bm{\lambda}), s_\delta^2(\bm{x}_c; \bm{\psi}_\delta, \bm{\lambda})) \\
        & \mathcal{E}(\boldsymbol{\lambda}) \thicksim \mathcal{N}(0, \sigma_{obs}^2(\bm{\lambda})) \\
    \end{split}
\label{eq:bc_data_generating_models_gaussian}
\end{equation}
As such, the data generating process $\mathcal{Y}_E$ under Gaussian formulation above is
\begin{equation}
	\begin{split}
		& \mathcal{Y}_E \thicksim \mathcal{N}(m_*, s^2_*) \\
		& m_*(\bm{x}_c, \hat{\bm{x}}_m; \bm{\psi}_m, \bm{\psi}_\delta, \bm{\lambda}) = m_M(\bm{x}_c, \hat{\bm{x}}_m; \bm{\psi}_m, \bm{\lambda}) + m_\delta(\bm{x}_c; \bm{\psi}_\delta, \bm{\lambda}) \\
		& s^2_*(\bm{x}_c, \hat{\bm{x}}_m; \bm{\psi}_m, \bm{\psi}_\delta, \sigma_{obs}^2, \bm{\lambda}) = s_M^2(\bm{x}_c, \hat{\bm{x}}_m; \bm{\psi}_{m}, \bm{\lambda}) + s_\delta^2(\bm{x}_c; \bm{\psi}_\delta, \bm{\lambda}) + \sigma_{obs}^2(\bm{\lambda})
	\end{split}
\label{eq:bc_data_model_gaussian}
\end{equation}
where $m_*$ and $s^2_*$ are the mean and the standard deviation of the experimental data generating process under Gaussian formulation, respectively.

% Generic Likelihood
Given a set of experimental data $\mathbf{y}$ taken at $\mathbf{x}_c$ and observed on an observation layout $\bm{\Lambda}$,
\marginpar{Likelihood function}
the likelihood function is then defined as follows
\begin{equation}
  \mathcal{L}(\hat{\bm{x}}_m, \bm{\psi}_\delta, \bm{\psi}_\epsilon; \mathbf{y}, \mathbf{x}_c, \bm{\Lambda}) \equiv p(y_E = \mathbf{y} | \bm{x}_c = \mathbf{x}_c, \hat{\bm{x}}_m, \bm{\psi}_\delta, \bm{\psi}_{\epsilon} ; \bm{\Lambda})
\label{eq:bc_likelihood}
\end{equation}
Under Gaussian formulation, the likelihood function is obtained by using Gaussian density of Eq.~(\ref{eq:bc_data_model_gaussian}) for $p$.
Note that if the set of experimental data is simultaneously given on the observation layout $\bm{\Lambda}$ then the covariance matrix $\Sigma_*$ is used instead of the standard deviation $s^2_*$,
\begin{equation}
	\begin{split}
	\Sigma_*(\bm{x}_c, \hat{\bm{x}}_m; \bm{\psi}_m, \bm{\psi}_\delta, \bm{\psi}_\epsilon, \bm{\Lambda}) = & \Sigma_M(\bm{x}_c, \hat{\bm{x}}_m; \bm{\psi}_{m}, \bm{\Lambda}) + \ldots \\ 
	& \Sigma_\delta(\bm{x}_c; \bm{\psi}_\delta, \bm{\Lambda}) + \Sigma_{obs}(\bm{\psi}_\epsilon; \bm{\Lambda})
	\end{split}
\label{eq:bc_gaussian_covariance_matrix}
\end{equation}
where $\Sigma_M$, $\Sigma_\delta$, and $\Sigma_{obs}$ are the $P \times P$ covariance matrices of the observation under their respective parametric kernels, with $P$ the dimension of the experimental data.

% Full probability model
According to the Bayes' theorem, the joint posterior probability of the model parameters $\bm{x}_m$ and
\marginpar{Joint posterior density}
the hyper-parameters associated with the model bias and the observation error is given as, 
\begin{equation}
	\begin{split}
  p(\hat{\bm{x}}_m, & \bm{\psi}_\delta, \bm{\psi}_{\epsilon_y} | \mathbf{y}, \mathbf{x}_c; \bm{\Lambda}) = \ldots \\
	& \frac{\mathcal{L}(\hat{\bm{x}}_m, \bm{\psi}_\delta, \bm{\psi}_{\epsilon_y} ; \mathbf{y}, \mathbf{x}_c, \bm{\Lambda}) \cdot p(\hat{\bm{x}}_m) \cdot p(\bm{\psi}_{\epsilon_y}; \bm{\Lambda}) \cdot p(\bm{\psi}_{\delta}; \bm{\Lambda})}{p(y_E = \mathbf{y} | \bm{x}_c = \mathbf{x}_c ; \bm{\Lambda})}
	\end{split}
\label{eq:bc_joint_posterior}
\end{equation}
where $p(\hat{\bm{x}}_m)$, $p(\psi_{\bm{\epsilon}_y}; \bm{\Lambda})$, and $p(\bm{\psi}_{\delta}; \bm{\Lambda})$ are the prior probabilities for the model parameters, the model bias hyper-parameters, and the observation error hyper-parameters, respectively.

The denominator of the Eq.~(\ref{eq:bc_joint_posterior}) is a normalizing constant with respect to the model parameters and the hyper-parameters such that Eq.~(\ref{eq:bc_joint_posterior}) is a valid probability density (i.e., integration over the domain yields the value $1.0$).
\marginpar{Normalizing constant}
As such, it is defined as a multidimensional integral of the following,
\begin{equation}
	\begin{split}
	p(y_E = \mathbf{y} | \bm{x}_c = \mathbf{x}_c ; \bm{\Lambda}) = & \int \mathcal{L}(\hat{\bm{x}}_m,\bm{\psi}_\delta, \bm{\psi}_\epsilon ;  \mathbf{y}, \mathbf{x}_c, \bm{\Lambda}) \cdot \ldots \\
	& p(\hat{\bm{x}}_m) \cdot p(\bm{\psi}_\epsilon; \bm{\Lambda}) \cdot p(\bm{\psi}_\delta; \bm{\Lambda}) \, d\hat{\bm{x}}_m d\bm{\psi}_\epsilon d\bm{\psi}_\delta
	\end{split}
\label{eq:bc_normalizing_constant}
\end{equation}

The specifications of the likelihood and the associated priors completely specify the Bayesian statistical calibration framework for the model parameters.
An inference of the model parameters can be made based on the resulting probability model through a simulation method outlined in Section~\ref{sec:bc_mcmc}.
However, besides the model parameters and controllable inputs, the complete formulation above also involves additional parameters associated with the statistical models: $\bm{\psi}_\delta$, $\bm{\psi}_{obs}$, and $\bm{\psi}_M$ the hyper-parameters for the model bias, the observation error, and the simulator approximation (if apply), respectively.
In principle, they are now also part of the calibration problem, increasing the size (in dimension) and the complexity of it.
To simplify the problem, a modularized approach is taken in this thesis as briefly explained in the following section.

%---------------------------------------------------------------------------------
\subsection{Modularization of the Bayesian Framework}\label{sub:bc_modularization}
%---------------------------------------------------------------------------------

%The actual forms of the densities in 
%then, under the additive formulation, the data generating process for $\mathcal{Y}_E$ can be obtained by adding all the three terms above.

\newpage
%*******************************************************************
\section{\glsfirst[hyper=false]{mcmc} Simulation}\label{sec:bc_mcmc}
%*******************************************************************

% Introductory Paragraph
The formulation of Bayesian calibration of a computer model presented above results in a joint posterior \gls[hyper=false]{pdf} for all the parameters involved.
\marginpar{Posterior uncertainty of the model parameters}
This density contains all the information (and consequently, the uncertainties) regarding the model parameters conditioned on the observed data and the assumed data-generating process.
The uncertainties associated with the model parameters can then be represented using different summary statistics, many of which involve integration.

For example,
the uncertainties associated with a model parameter $x_d$ can be represented with its variance,
which is defined as
\begin{equation*}
	\begin{split}
	\mathbb{V}[\mathcal{X}_d] & = \mathbb{E}[\mathcal{X}_d^2] - \mathbb{E}^2[\mathcal{X}_d]\\
	                          & = \int_{-\infty}^{\infty} x_d^2 p(\bm{x} | \mathbf{y}) d\bm{x} - \left( \int_{-\infty}^{\infty} x_d p(\bm{x} | \mathbf{y}) d\bm{x} \right)^2
	\end{split}
%\label{eq:variance_marginalization}
\end{equation*}
where the integrations are carried out over the support of $p(\bm{x}|\mathbf{y})$, assumed to be on the entire real line.
An alternative way to summarize the uncertainties of a model parameter is through its $\theta$-quantile $Q^{\theta}_d$,
which for parameter $x_d$ is defined as
\begin{equation*}
	Q^{\theta}_d \, : \, \mathbb{P} (\mathcal{X}_d \leq Q^{\theta}_d) = \int^{Q^{\theta}_d}_{-\infty} \int\limits_{\mathbf{X}_{\sim d}} p(x_d, \bm{x}_{\sim d} | \mathbf{y}) d\bm{x}_{\sim d} d x_d = \theta
%	\label{eq:quantile_integral}
\end{equation*}
Assuming that the support of $p(\bm{x}|\mathbf{y})$ is on the entire real line.
In this manner,
the $95\%$ confidence interval of the parameter is written as $Q_d^{0.025} \leq \mathcal{X}_d \leq Q_d^{0.975}$.

Though these summaries might be of interest themselves,
\marginpar{Posterior uncertainty of the model prediction}
in an application setting, the model parameters uncertainties are often propagated through the simulation model to obtain the uncertainty in the output (prediction).
Hence, the output from a simulation model $y = f(\bm{x})$ is expressed as random variable $\mathcal{Y}$ from a transformation of random variable $\bm{\mathcal{X}}|\mathbf{y}$ by the function $f$
\begin{equation*}
	\mathcal{Y} = f(\bm{\mathcal{X}} | \mathbf{y})  \, ; \, p_{\bm{\mathcal{X}} | \mathbf{y}} (\bm{x}) = p(\bm{x}| \mathbf{y})  
\end{equation*}
where the \gls[hyper=false]{pdf} of $\bm{\mathcal{X}} | \mathbf{y}$ is the posterior density $p(\bm{x}| \mathbf{y})$.
The actual \gls[hyper=false]{pdf} of $\mathcal{Y}$ follows the rule of transformation of random variable
and it represents the uncertainty in the output due to the uncertainty in the input parameter conditioned on the data.
This uncertainty can also be summarized with various statistics and, as before, many of which involve integration operation.
For instance, the variance of the output, is written as
\begin{equation*}
	\mathbb{V}[\mathcal{Y}] = \int_{-\infty}^{\infty} f^2(\bm{x}) p(\bm{x} | \mathbf{y}) d\bm{x} - \left( \int_{-\infty}^{\infty} f(\bm{x}) p(\bm{x} | \mathbf{y}) d\bm{x} \right)^2
\end{equation*}

The posterior density $p(\bm{x}|\mathbf{y})$ and the function $f(\bm{x})$, however, are in practice highly multidimensional functions and 
the performance of numerical integration is typically worsened with an increasing number of dimensions of the input parameter space.
\marginpar{Challenges in dealing with posterior density}
At the same time,
conducting \gls[hyper=false]{mc} simulation for estimating the integrals (such was done in Chapter~\ref{ch:gsa}) is not straightforward in this case.
The multiplication of likelihood and prior density will, in general, yield an arbitrary posterior density not available in a closed-form expression.
As a result, generating independent samples from the posterior density required for the \gls[hyper=false]{mc} estimation becomes a difficult task.

This section presents a simulation approach, the so-called \gls[hyper=false]{mcmc} simulation,
to directly generate samples from an arbitrary \gls[hyper=false]{pdf}. 
These samples, in turn, are useful for estimating various quantities given as examples above independent on the dimension of the input parameter space.

Although in the context of Bayesian data analysis such an arbitrary \gls[hyper=false]{pdf} of interest is the posterior \gls[hyper=false]{pdf},
the problem of generating sample from an arbitrary \gls[hyper=false]{pdf} is quite general.
As such, in the following discussion, a generic notation for an arbitrary \gls[hyper=false]{pdf} $p(\bm{x})$ is used instead of $p(\bm{x}|\mathbf{y})$.

%----------------------------------------------------
\subsection{Motivation}\label{sub:bc_mcmc_motivation}
%----------------------------------------------------

% Introductory Paragraph
Consider the following problem: Given a \gls[hyper=false]{pdf} $p:\mathbf{X} \subseteq \mathbb{R}^D \mapsto \mathbb{R}^+$,
generate a set of samples $\{\bm{x}_n\}_{n=1}^N$ from the \gls[hyper=false]{pdf}.
\marginpar{Problem statement}
It is assumed that the \gls[hyper=false]{pdf} can be evaluated at any given $\bm{x} \in \mathbf{X}$, at least up to a proportionality constant:
\begin{equation}
	p(\bm{x}) = \frac{p^*(\bm{x})}{C} \propto p^*(\bm{x})  
\label{eq:bc_prop_post}
\end{equation}
The proportionality constant in the above equation is the normalizing constant such that $p$ is a valid \gls[hyper=false]{pdf},
\begin{equation}
	C = \int p^*(\bm{x}) d\bm{x}  \Rightarrow \int p(\bm{x}) d\bm{x} = 1.0
\label{eq:bc_prop_const}
\end{equation}
Such samples can then be used, among other things,
to evaluate different summary statistics (such as expectation, variance, etc.) of $\bm{x}$ itself or
of any function under the \gls[hyper=false]{pdf}.

% Why it it difficult to samples
Generating samples from an arbitrary multidimensional density function is generally a difficult task.
\marginpar{A correct sampling}
Intuitively, for a given sample size, correctly generating samples from a density means
that the sample values have to be distributed proportional to its \gls[hyper=false]{pdf}.
There should be more samples in the region where the \gls[hyper=false]{pdf} value is high,
and less in the the region where the \gls[hyper=false]{pdf} value is low.
For a complex multidimensional density function,
these locations are not known a priori and might have to be identified exhaustively \cite{Mackay2005}.

% Inverse Transform Sampling and Its Problem
In a relatively low dimension,
the most common way of generating sample from a given density is
by inverse transform sampling coupled with a random number generator.
\marginpar{Inverse transform sampling}
The approach requires the quantile function of the \gls[hyper=false]{pdf}.
To obtain the quantile function,
the density has to be integrated and its normalizing constant has to be computed.
Appendix~\ref{app:its} provides a more detail account on the topic.
Many univariate (and some multivariate) \glspl[hyper=false]{pdf} are widely studied and the analytical solutions to their quantiles are available \cite{Lange2010}.
Additionally, sampling algorithms readily exist for several multivariate densities due to their special properties (e.g., generating sample from a multivariate normal density in Appendix~\ref{app:mvn_sampling}).
However, this will not be the case for an arbitrary density function of higher dimension.
In the absence of an analytical solution,
the cost of numerically integrating such function might itself be very costly due to the size of the dimension.

% Illustration
To illustrate this point,
\marginpar{Illustration}
consider the following bivariate (unnormalized) density parameterized by location parameters $\mu_1, \mu_2$and scale parameters $\sigma_1, \sigma_2$:
\begin{equation}
	\begin{split}
	& p^*(x_1, x_2) = \frac{\exp{(-(x_1 - \mu_1)/\sigma_1)} \exp{(-(x_2 - \mu_2)/\sigma_2)}}{(1 + \exp{(-(x_1 - \mu_1)/\sigma_1)} + \exp{(-(x_2 - \mu_2)/\sigma_2)})^3} \, \\ 
	& x_1, x_2 \in \mathbb{R}; \mu_1, \mu_2 \in \mathbb{R};\, \text{and} \, \sigma_1, \sigma_2 \in \mathbb{R}^+
	\end{split}
\label{eq:bc_unnormalized_gumbel}
\end{equation}
In this simple example,
the inverse transform sampling works well as the function is relatively easy to integrate in order to obtain the quantile function.
However, for the sake of illustration, it is assumed here that the quantile function of Eq.~(\ref{eq:bc_unnormalized_gumbel}) and its numerical approximation are not available.
Fig.~\ref{fig:ch5_plot_gumbel_illustration} shows the contour plot of the joint density as well as the marginal density for each of the variate.
\normdoublefigure[pos=tbhp,
                  mainlabel={fig:ch5_plot_gumbel_illustration},
                  maincaption={Joint and marginal densities plots of the unnormalized \gls[hyper=false]{pdf} in the example. The parameters used in the example are: $\mu_1 = 5, \mu_2 = 2, \sigma_1 = 1.25$, and $\sigma_2 = 3$.},%
									mainshortcaption={Joint and marginal densities plots for the unnormalized \gls[hyper=false]{pdf} in the example.},
                  leftopt={width=0.45\textwidth},
                  leftlabel={fig:ch5_plot_gumbel_illustration_1},
                  leftcaption={Joint density},
                  %leftshortcaption={},%
                  rightopt={width=0.45\textwidth},
                  rightlabel={fig:ch5_plot_gumbel_illustration_2},
                  rightcaption={Marginal density},
                  %rightshortcaption={},
                  %spacing={\hfill}
                 ]
{../figures/chapter5/figures/plotGumbelIllustration_1}
{../figures/chapter5/figures/plotGumbelIllustration_2}

% Grid Approach
A straightforward approach to generate samples from a given multivariate density is done by first discretizing the input parameter space of the density function and evaluate the density at the discretized points.
\marginpar{Discretized grid approach}
Supposed the domain of the density has been discretized uniformly in each dimension with a level $\Delta$ resulting in $\{\bm{x}_i\}_{i=1}^{I}$ with $I$ the number of discretized points.
At the discretized levels, each value of $\bm{x}_i$ is associated with the probability $p(\bm{x}_i) = p^*(\bm{x}_i) / \sum_i p^*(\bm{x}_i)$.
These pairs constitute a complete discrete probability distributions.
Generating samples from such a discrete probability distribution is simple in modern computing environment \cite{Mackay2005}.

% The example
Fig.~\ref{fig:ch5_plot_gumbel_sample_grid} illustrates this procedure for the example given above.
First, the input parameter space is windowed in $\mathbf{X}\in[-25,25]^2$ before being discretized in $\Delta = 50$ levels.
\marginpar{Discretized grid approach illustrated}
This results in $(\Delta + 1)^2 = 2'601$ discretized points at which the density is evaluated (Fig.~\ref{fig:ch5_plot_gumbel_sample_grid_1}).
Next, the density values are taken to be the probability for each of the $2'601$ discretized points.
Together they make up a complete discrete probability distribution from which samples can be readily generated.

% The figure explained
Fig.~\ref{fig:ch5_plot_gumbel_sample_grid_1} shows $5'000$ samples generated from the discrete distribution.
Darker points indicate that the values have been sampled multiple times following the actual underlying \gls[hyper=false]{pdf}.
The contour of the analytical joint density is overlaid to serve as a guide. 
Figs.~\ref{fig:ch5_plot_gumbel_sample_grid_2} and~\ref{fig:ch5_plot_gumbel_sample_grid_3} show the histograms for each of the marginals.
The figure shows that the generated samples are indeed approximately distributed as the given \gls[hyper=false]{pdf}.
\bigtriplefigure[pos=tbhp,
								 mainlabel={fig:ch5_plot_gumbel_sample_grid},
			           maincaption={Sampling from a multivariate density by discretizing the input parameter space in grids. The input parameter space is discretized into $\Delta = 50$ levels. The density is then evaluated at the discretized points. (Left) $5'000$ samples are generated following the resulting discrete probability distribution; (Center and Right) The histograms of the marginals approximately follow the shape of the respective analytical marginal density. The marginal densities have been normalized to match the peak of the histogram.},
			           mainshortcaption={Sampling from a multivariate density by discretizing the input parameter space in grids.},%
			           leftopt={width=0.30\textwidth},
			           leftlabel={fig:ch5_plot_gumbel_sample_grid_1},
			           leftcaption={Joint samples},
			           midopt={width=0.30\textwidth},
			           midlabel={fig:ch5_plot_gumbel_sample_grid_2},
			           midcaption={Marginal of $x_1$},
			           rightopt={width=0.30\textwidth},
			           rightlabel={fig:ch5_plot_gumbel_sample_grid_3},
			           rightcaption={Marginal of $x_2$},
			           spacing={},
			           spacingtwo={}]
{../figures/chapter5/figures/plotGumbelSampleGrid_1}
{../figures/chapter5/figures/plotGumbelSampleGrid_2}
{../figures/chapter5/figures/plotGumbelSampleGrid_3}

% Problem with Discretized Grid
The main issue with the discretized grid approach, conceptually simple as it is,
is the curse of dimensionality similar to the one mentioned in the previous chapters.
\marginpar{curse of dimensionality}
The number of density evaluations grows exponentially with the number of dimension.
As a rule, for a given discretization level $\Delta$ and dimension $D$,
the number of density evaluations is $(\Delta+1)^D$.

% In relation to Bayesian data analysis
In the context of Bayesian data analysis,
the multidimensional likelihood function inside the posterior generally can be a complex function.
In consequence, a very fine discretization level might be required to appropriately capture the function behavior at important regions which are unknown a priori.
On top of that, the computational cost of evaluating the (complex) posterior density becomes nonnegligible.
As such, except for a very simple likelihood function and/or in a very low dimension,
grid approach is deemed inapplicable for generating samples from an arbitrary multidimensional density.

% Entry to Markov Chain
To circumvent these issues,
a sampling technique based on the theory of stochastic process is widely adopted.
Specifically, by constructing a Markov chain\footnote{a realization of a Markov process} of the input parameters values,
the resulting process will eventually becomes stationary and simultaneously converge to any given \gls[hyper=false]{pdf} (i.e., \emph{target density}).
In other words, the samples generated from the stationary process are distributed according to the given density. 
Theoretically, this family of techniques would be independent of the dimension of the input parameter space and its convergence are guaranteed.

The next subsection briefly presents the basics of Markov chain and its importance in solving the the problem of generating samples from an arbitrary density.
Afterward,
a traditional and a more recent methods for constructing a Markov chain for the purpose of \gls[hyper=false]{mc} simulation are introduced. 

%----------------------------------------------
\subsection{Markov Chain}\label{sub:bc_mcmc_mc}
%----------------------------------------------
%once more the posterior \gls[hyper=false]{pdf} of the model parameters conditioned on the observed data.
%Strictly speaking such a \gls[hyper=false]{pdf} will also be conditioned on a selected data-generating process model $\mathcal{M}$, i.e., $p(\bm{x}|\mathbf{y},\mathcal{M})$.
%However, in the following discussion, this conditioning is implicitly assumed and removed from the notation yielding
%\begin{equation}
%	p(\bm{x} | \mathbf{y}) = \frac{p(\bm{y} = \mathbf{y} | \bm{x}) p(\bm{x})}{\int p(\mathbf{y} | \bm{x}) p(\bm{x}) d\bm{x}}
%\label{eq:pdf_posterior}
%\end{equation}
% Introductory paragraph (Why Markov Chain)

% Markov Chain Definition

% Theorem, Markov Chain and Stationary Distribution

%------------------------------------------------------------
\subsection{Markov Chain Monte Carlo}\label{sub:bc_mcmc_mcmc}
%------------------------------------------------------------

% Introductory Paragraph

% Metropolis-Hastings Algorithm

% Simple Example

% Different Samplers

%---------------------------------------------------------------------
\subsection{Affine-Invariant Ensemble Sampler}\label{sub:bc_mcmc_aies}
%---------------------------------------------------------------------

% Introductory Paragraph

% Affine-Invariant Ensemble Sampler, Motivation

% Affine-Invariant

% Ensemble Sampler

% AIES Algorithm

% Implementation

% Possible problem and Why not
%******************************************************************************************************************************
\section[Diagnosing Convergence]{Diagnosing Convergence of an \gls[hyper=false]{mcmc} Simulation}\label{sec:bc_mcmc_diagnostic}
%******************************************************************************************************************************

% With a minimal tuning to deal with in AIES algorithm, we are left with the problem of assessing convergence.
% In particular two key issues persists
% The previous discussion spoke rarely about the condition.
% It was demonstrated, through graphical representations the main idea of convergence in distributin.
% This section seeks to answer, the following two important questions of Markov chain 
% For an arbitrary starting distribution, when the chain can be considered stationary
% It is not to say that these initial part of the chain is not part of the target distribution support.
% They are, but as a distribution they are heavily biased.
% Assuming a stationary distribution have been reached, how long the chain should be run to meet certain level of accuracy
%*************************************************************************************************************
\chapter[Introduction]{Quantifying Uncertainty of Computer Model: Forward and Backward}\label{ch:introduction}
%*************************************************************************************************************

Lorem Ipsum

%************************************************************************************************************
\section{Computer Simulation and Safety Analysis of Nuclear Power Plant}\label{sec:intro_computer_simulation}
%************************************************************************************************************

The ubiquity of computer simulation applications in many fields of science and engineering results in an even more pervasive definitions of the term \textit{scientific computer simulation} itself 
and other associated terms such as \textit{model} and \textit{simulation}.
\marginpar{scientific computer simulation}
Though most of the definitions in the literature are not necessarily in contradiction to each other, 
to avoid confusion, this thesis adopts a recent definition proposed by Kaizer et al.\cite{Kaizer2015} quoted below,

\begin{quote}
	Scientific Computer Simulation is the imitation of a behavior of a system, entity, phenomenon, or process in the physical universe 
	using limited mathematical concepts, symbols, and relations through the exercise or use of scientific computer model.
\end{quote}

There are three main points highlighted in this definition.
\marginpar{model, simulation, and computer simulation}
First, this definition accentuates the difference between a \emph{model} and its \emph{simulation}.
Specifically, the former deals with the notion of representation, while the latter deals with the notion of imitation of a behavior.
Secondly, what makes a model scientific is that it treats physical phenomena or the behavior of a real world system as its subject.
Thirdly and finally, the modifier \emph{computer} in the definition makes it explicit that digital computer is used to solve whatever mathematical models serve as the representation.
This is usually the case for mathematical models that cannot be solved analytically.
Though this limitation what makes a solution of the model possible in the first place, 
it also affects the solution and its possible interpretation and thus many computational-related aspects also need to be comprehensively considered.
\footnote{Beven \cite{Beven2009} articulates this distinction into three levels of model: perceptual model (i.e., theoretical description of some physical phenomena), formal model (i.e., the mathematical description of it), and procedural model  (i.e., computer implementation of the formal model). For many physical system modeling applications, only the procedural model is able to make a quantitative prediction of the system. These distinctions are useful in acknowledging the level of approximation involved in moving from perceptual to procedural model.}
\section{Computer Experiment}\label{sec:intro_computer_experiment}

Granted, a simplicistic model cannot be expected to imitate all the important features of a complex physical phenomenon.
And yet, there is a tendency of developing and applying overly complex model, with numerous parameters and multiple non-linear relationships, for computer simulation.
This tendency has invited many critics over the year (citation needed).
In nuclear science and engineering, for instance, Zuber \cite{Zuber2001} and Wullf \cite{Wulff2007} have long critized the development and the use of multi-fluid model for thermal-hydraulics simulation as high in complexity and in maintenance cost, but low in fidelity and its usefulness.

The goal of computer simulation, complex or otherwise, is to provide prediction.
The main characteristic (and source of criticism) of using multi-parameter complex model for simulation is that the the relationship between numerous inputs and outputs becomes increasingly opaque.
The impact of changing one parameter alone, or especially together, on the prediction is hard to disentangle or intuit.
Furthermore, as appropriate values of inputs might not be fully known, they are often given simply over range of interest containing different possible permissible values.
It is thus seldom the case that one single simulation is sufficient to provide a reliable answer according to hardly any objective of computer simulation.
As analytical solution
%*************************************************************************************************************************
\section{Uncertainty Quantification in Nuclear Engineering Thermal-Hydraulics}\label{sec:intro_uncertainty_quantification}
%*************************************************************************************************************************

% Introductory Paragraph
Before continuing the discussion of uncertainty analysis of code predictions, it will be worthwhile to define some additional terminologies to avoid later confusion.

In making a connection with the notion of \emph{simulator} introduced in Section~\ref{sec:intro_computer_simulation}, 
recall that from Fig.~\ref{fig:ch1_th_system_code} an \emph{input deck} is distinct from the code itself.
Fig.~\ref{fig:ch1_simulator_io} depicts the notion of simulator of a thermal-hydraulics system in a more generic way, as an input/output model.
\begin{figure}[bth]	
	\centering
	\includegraphics[width=\textwidth]{../figures/chapter1/figures/simulator_io}
	\caption[Simplified illustration of a simulator as an input/output model.]{Simplified illustration of a simulator as an input/output model.}
	\label{fig:ch1_simulator_io}
\end{figure}

Indeed input deck defines specific problem (i.e., system) of interest.
It includes the specifications for geometrical configuration (i.e., the nodalization), choice of material and fluid involves, as well as initial and boundary conditions.
It may also include the setting for the numerical solver.
Some of those specifications (such as the boundary conditions, etc.) are parametrized and constitutes \emph{controllable inputs} denoted by $\bm{x}_c$\footnote{later on, \emph{controllable} inputs corresponds to the equivalent parameters in a physical experiment which can be controlled by the experimentalist.}.
The conservation equations of the code are closed with additional set of closure laws (and other sub-models) $\mathcal{M}_i(\bm{x}_c, \bm{x}_m, \bm{u})$.
These closure laws are, in turn, parametrized with a set of model specific parameters denoted by $\bm{x}_m$ which is referred to as the \emph{physical model parameters}.
 
Specifying the input deck, as far as user is concerned, completely defines the problem and the code will solve the conservation equations which output the dynamic state of relevant physical variables $\mathbf{u}(\bm{r}, t)$ (e.g., fluid pressure, temperature, wall temperature, etc.).
In practice, these raw simulation outputs are further post-processed to obtain the so-called \glspl[hyper=false]{qoi} the are relevant to the problem at hand.

%--------------------------------------------------------------------------
\subsection{Forward Uncertainty Quantification}\label{sub:intro_uq_forward}
%--------------------------------------------------------------------------

% Best-estimate, limitation
As explained, best-estimate analysis uses more realistic modeling assumptions for analyzing transient behavior of \gls[hyper=false]{npp}.
It attempts as realistically as possible to describe the behavior of the relevant physical processes occur during the plant transient.
And yet, even the best available understanding of the physical process is still limited.
Understanding of complex phenomena might not yet adequate and data support for some processes can be very limited.s
Simplifying assumptions, approximations, and expert judgments to some degree are unavoidable and still required to have a complete analysis.

% Best-estimate, plus uncertainty
Hence, best-estimate analysis has to be complemented with uncertainty analysis.
\marginpar{Best-estimate plus uncertainty}
The ultimate goal of uncertainty analysis is to associate code prediction with its uncertainty.
These combined quantities are then compared with certain regulator safety limits (e.g., \gls[hyper=false]{pct}) to check whether the limits still fall outside the uncertainty band of the code prediction.

% Source of possible uncertainties
There are several known sources of uncertainty that render the prediction on $\bm{u}(\mathbf{r},t)$ and its derived quantities uncertain.
The following are the sources of primary interest in the present research:
\begin{enumerate}
	\item 1 
  \item 2
  \item 3
\end{enumerate}

% Forward uncertainty quantification, Inputs as random variables
In statistical uncertainty analysis, the controllable inputs and physical model parameters are modeled as random variables ($\bm{\mathcal{X}}_c$ and $\bm{\mathcal{X}}_m$, respectively) equipped with probability distributions.
\marginpar{Forward uncertainty quantification}
Then using \gls[hyper=false]{mc} technique, samples are generated from their respective distributions and they are used to run the code multiple times.
Finally, statistics of code outputs (raw or post-processed), are summarized to obtain the uncertainty measure of the prediction.
In other words, the uncertainties in the controllable inputs and physical model parameters are \emph{propagated forward} through the code to quantify the uncertainty of the predictions as shown in Fig.~\ref{fig:ch1_simulator_uq_forward}.
The practice of propagating uncertainty by \gls[hyper=false]{mc} is widely accepted in the nuclear engineering thermal-hydraulics community \cite{Lellouche1990,Glaeser1994,Glaeser2008}.
\begin{figure}[!bth]	
	\centering
	\includegraphics[width=\textwidth]{../figures/chapter1/figures/simulator_uq_forward}
	\caption[Simplified flowchart of forward uncertainty quantification of a simulator prediction.]{Simplified flowchart of forward uncertainty quantification of a simulator prediction. Notice that the simulator has been parametrized by the controllable inputs and physical model parameters, each of which are represented as random variable.}
	\label{fig:ch1_simulator_uq_forward}
\end{figure}

% Source of uncertainty, initial and boundary condition

% Source of uncertainty, physical model parameters

% Model parameters
The physical model parameters, however, are conceptually different.
\marginpar{Model parameters}
The physical models referred to in this thesis are usually represented either in the form of correlation or phenomenological (mechanistic) model.
The parameters associated with these models are derived from experimental data.
They can either represent physically meaningful quantities (e.g., reaction rate coefficient) or not (i.e., a tuning parameter).
In either way, there are uncertainties associated with these parameters especially as the conditions of their intended use is making prediction can be (at times, very) different than the conditions of their derivation.
The conceptual distinction related to model parameters will be revisited in Chapter~\ref{ch:bayesian_calibration}.

%--------------------------------------------------------------------------
\subsection{Inverse Uncertainty Quantification}\label{sub:intro_uq_inverse}
%--------------------------------------------------------------------------

% Introductory paragraph, motivation
The physical model parameters often cannot be measured directly.
In this situation, in order to estimate their values,
experiments with well-specified conditions are carried out and correlations (models) are fitted based on the experimental data. 
Recall that from the discussion of closure laws origin in Section~\ref{sub:intro_th_system_code},
this can also apply to phenomenological models where free parameters are allowed to be tuned according to match the experimental data.
Ultimately, the optimal value of the estimated parameter is implemented in the code.
As the code is used to simulate multiple phenomena during a transient, additional experiments are carried out in larger, more complex test facilities.
Finally, based on the data obtained, the models can be validated.

% Inverse uncertainty
To obtain the uncertainty associated with the model parameters obtained in the manner above, the problem can be posed as an inverse problem.
In this setting, given a set of experimental data $\{\mathbf{D}\}$ with a known controllable inputs $\mathbf{x}_c$, the task is then to infer the value of the physical model parameters.
It is important to acknowledge various sources of uncertainty previously mentioned:
experimental data and controllable inputs are observed but perhaps there remains residual uncertainty and observation error associated with them;
and the associated models are also only an approximation of the real physical processes with a possible, but unknown, systematic bias.
\bigfigure[pos=tbhp,
           opt={width=1.0\textwidth},
           label={fig:ch1_simulator_uq_inverse},
           shortcaption={Simplified flowchart of inverse uncertainty quantification of model parameters.}]
{../figures/chapter1/figures/simulator_uq_inverse}
{Simplified flowchart of inverse quantification for model parameters of a simulator.}

% Connection to PREMIUM Benchmark
The importance of characterizing the uncertainty in the physical models parameters was acknowledged by the \gls[hyper=false]{wgama} of the \gls[hyper=false]{oecd}/\gls[hyper=false]{nea}.
This led to the \gls[hyper=false]{premium} project.
Its main goal is to report the state-of-the-art of the available methodologies to quantify the uncertainty in the physical models parameters.
The following will briefly describe the project and highlight its main findings.

%-------------------------------------------------------------
\subsection{OECD/NEA PREMIUM project}\label{sub:intro_premium}
%-------------------------------------------------------------

% Introductory paragraph
The \gls[hyper=false]{premium} project was an activity launched by the \gls[hyper=false]{oecd}/\gls[hyper=false]{nea} in $2012$ and concluded in $2016$ with the aim to advance the methods for quantifying the uncertainties associated with the physical model parameters in \gls[hyper=false]{th} system codes.
It was the continuation of the previous project \gls[hyper=false]{bemuse}, which concetrated on the propagation and sensitivity analysis of the input uncertainties in large scale simulation (large break \gls[hyper=false]{loca}).
The main finding of \gls[hyper=false]{bemuse} can be found in \cite{Perez2011}.
The emphasis of the \gls[hyper=false]{premium} benchmark was placed on the derivation of the model parameters uncertainty and their validation.

% Scope of the Project

% Main Findings

\input{Chapters/1_introduction/premium}
\input{Chapters/1_introduction/bayesian_framework}
%*********************************************************************************
\section{Objectives and Scope of the Thesis}\label{sec:intro_objectives_and_scope}
%********************************************************************************* 

With a larger context provided above,
this section presents briefly and specifically the statement of the problem,
followed by the objectives as well as the scope of the present doctoral research.

%--------------------------------------------------------------------------
\subsection{Statement of the Problem}\label{sub:intro_statement_of_problem}
%--------------------------------------------------------------------------

% Introductory Paragraph
The development of closure laws for reflooding described in \cite{Nelson1992,USNRC2012} showed the difficulties and the amount of assumptions used.
In a nutshell, system code development is an effort to consolidate correlations and mechanistic models, to create a phenomenological-based simulation code that can provide best-estimate results.
This consolidated effort results in a code that can simulate wide range of transients foreseen in nuclear power plant operation in a best-estimate manner.
Alas, to come up with a consistent set of closure laws is a great challenge for code developers.

% Closure Laws Difficulty, Conceptual
The closure laws required to close the two-fluid model pose particularly difficult challenges \cite{Wulff2007}.
For instance, to have a correlation of heat transfer between the wall and the fluid, temperature data from each of the constituents are needed (i.e., the wall, the liquid phase, and the gas phase).
But measuring temperature of the individual phases in an arbitrary interfacial topology has its own technical difficulties to the extend that no such data exists or available to be implemented in the closure laws.
Additionally, the experiments to obtain hydrodynamic closure laws (e.g., interfacial friction factor, wall friction factor, etc.) were generally carried out in adiabatic conditions.
As a result, this excludes the coupling of any heat transfer phenomena between the phases and the wall in such correlation.

% Closure Laws Difficult, Practical
Furthermore, during the development of a simulation code, programming considerations also came into the picture.
For robustness, simplification is often required and continuity is enforced.
Transitionary flow regime between two known (observed) flow regimes for which experimental data is not available is modeled to be the average of the two bounding regimes.
Different code development, which used different assumptions and experimental database, comes up with different set of closure laws with their own parametrization (see for instance \cite{Nelson1992} for TRAC code and \cite{Bestion1990} for CATHARE code).
Several authors have expressed their concerns about the uncertainty stemming from the closure laws \cite{Wulff2007,Petruzzi2008a,DAuria2012}.

% an Illustration
As an example of the point given above, consider that in the \gls[hyper=false]{trace} code, after some derivations the interfacial drag coefficient closure law in the inverted slug flow regime $C_{i,\text{IS}}$ is given by,
\begin{equation*}
	C_{i,\text{IS}} = \hat{x}_{m,\text{SET}} \times \frac{1}{24} \frac{\rho_g}{\text{La}} \frac{(1-\alpha)}{\alpha^{1.8}} \,\,\,;\,\,\, \hat{x}_{m,\text{SET}} = 0.75 
\label{eq:intf_drag_isf}
\end{equation*}
where $\rho_g$ is the density of the gas phase;
$\text{La}$ is the Laplace number;
$\alpha$ is the void fraction;
and $\hat{x}_{m,\text{SET}}$ is a fitting parameter.

There are several remarks about the closure law given above.
First, the second term in the right-hand side was derived from experimental data but not directly.
In the inverted slug regime, saturated liquid core breaks up into ligaments.
These ligaments are \emph{assumed} to take form as prolate ellipsoid.
The drag coefficient of distorted droplet experimental database is then \emph{assumed}.
Then to take into account the multi-particle effect, the coefficient is divided by the void fraction $\alpha$ raised to the power of $1.8$ (this, in turn, was taken from experimental data of inertial regime).
Lastly, the first term of the equation, $\hat{x}_{m,\text{SET}} = 0.75$ was put \emph{to match}, \emph{to calibrate against} the experimental data from the FLECHT-SEASET reflood facility.
This first term, although clearly \emph{non-physical}, is an important tuning parameter of the model nevertheless.
Its uncertainty should be considered in uncertainty analysis, especially when reflood is expected to occur.
Yet, no statement regarding the associated uncertainty is given.
Several other such terms exist \cite{USNRC2012}. 

% Statement of Problem
As illustrated above, it is clear that models in thermal-hydraulics system code, to a certain extent, flawed.
Various experimental programs were carried out to gain better understanding of important phenomena,
and to validate (and, as noted above, to calibrate) the models.
Series of the experiments, carried out in \glspl[hyper=false]{setf} were aimed to reproduce limited part of the transient in a selected component following a postulated scenario.
For example, in the case of reflooding, several facilities existed and data were available (FEBA, PERICLES, etc.).
But, there has not been an orchestrated effort to incorporate the accumulated data into the calibration process of the physical models, in a systematic way, while acknowledging multiple sources of the uncertainty in the process.

%--------------------------------------------------
\subsection{Objectives}\label{sub:intro_objectives}
%--------------------------------------------------

% Introductory (Overall Objective)
The purpose of the doctoral research is to quantify the uncertainty of physical model parameters
implemented in a thermal-hydraulics system code.
The physical models of interest describe the phasic interactions in a complex multiphase flow during a reactor transient, namely heat, mass, and momentum exchanges between vapor, water and structures.
These models are parametrized by physical or empirical tuning parameters, the values of which are uncertain.
This results in uncertain code prediction of important safety quantities, such as the evolution of the fuel cladding temperature during a postulated reactor transient.

Adopting probabilistic framework to conform to the statistical uncertainty propagation widely
adopted in the field of nuclear engineering, the uncertainties in the parameters are represented in
form of probabilistic density functions or their approximation.
The derivation of these functions is posed as an inverse statistical problem following Bayesian framework as the parameters themselves are not directly observable.
The doctoral research thus aims to present a consistent set of strategies in deriving the uncertainty of such model parameters based on experimental data.
This is done by consolidating and adapting recent developments in the applied statistics literature:

% Aim 1 (Global sensitivity analysis)
\begin{enumerate}
	\item \emph{to analyze and to better understand} the inputs/outputs relationship in a computer simulation with uncertain inputs.
	This is aimed at answering the question whether the current physical model in thermal-hydraulics system code \gls[hyper=false]{trace} can be identified with the available experimental data from test facilities.
	In other words, how to select important parameters to be inferred.
	\Glsfirst[hyper=false]{gsa} methodologies can be used to assist in identifying which parameters can be calibrated using the available data.
	A test facility might have multiple types of data and although the information content might not be the same for the different types, it might be worthwhile to consider each one of them.
	Finally, for each of the different types,
	the analysis is also conducted on various derived \glspl[hyper=false]{qoi}, some of which explicitly consider the output as function.
	By doing so, it is hope that interesting model behavior with respect to its parameters perturbation can be revealed.

% Aim 2 (Statistical Metamodeling)
	\item \emph{to approximate} the inputs/outputs relationship in a complex computer simulation for a faster evaluation.
	The step is required as the statistical calibration method adopted in thesis is computationally expensive, requiring numerous code runs in the order of hundreds of thousands and beyond.
	This approximation is done through a \glsfirst[hyper=false]{gp} metamodel resulting in a statistical metamodel.
	The highly multivariate nature of the outputs (time- and space-dependent) is dealt by a dimension reduction technique.
	Build upon the results of previous step, only parameters that are identified to be influential are included in the construction of the metamodel.

% Aim 3 (Bayesian Calibration)
	\item \emph{to statistically calibrate} the physical model parameters against various relevant experimental data.
	The word \emph{to calibrate} carries a disparaging interpretation related to \emph{to tweak}.
	However, using a Bayesian statistical framework, the aim of calibration is extended to simultaneously quantify the uncertainty of the parameter estimation.
	The framework includes various sources of uncertainty which can be modeled using probabilistically, including the model bias term.
	At the end, the parameters of interest will be either in the form of distributions conditioned on the data or samples generated from such distributions which are useful in the uncertainty analysis of code prediction.

% Aim 4 (Extrapolation)
	\item \emph{to validate} the statistical calibration results against experimental data set not used in the calibration step.
	As calibration only conducted using experimental data of limited experimental conditions, it is important, at the minimum, to validate the proposed methods by demonstrating the applicability of the results to the simulation of the phenomena of the same facility in different experimental conditions. 

\end{enumerate}

%----------------------------------------
\subsection{Scope}\label{sub:intro_scope}
%----------------------------------------

% Introductory paragraph
Although the proposed set of strategies in this research can be applicable to the analysis and calibration of any system code physical model,
it is illustrated by its application on the models of particular importance during simulation of reflooding,
i.e., the so-called \gls[hyper=false]{postchf} flow regimes.
There are several reasons for this emphasis:
\begin{itemize}

	% Reason 1
	\item Reflooding is an important part in the simulation of \glspl[hyper=false]{npp} transient during \gls[hyper=false]{loca}.
	Modeling reflooding determines the appropriate representation of the dynamics of heat transfer phenomena during the effort to rewet an uncovered core.
	Of paramount interest is to estimate the time the rod can be expected to be rewet as well as the maximum temperature reached prior to rewet.
	Reflood is a transient with highly coupled hydrodynamic-heat-transfer effects and it challenges the assumption made on the implemented closure laws.
	Indeed several reflood experimental programs conducted in \glspl[hyper=false]{setf} existed and were designed to validate reflood models in system code.
	Unfortunately, no orchestrated effort was done so far to consolidate the generated data in general and into \gls[hyper=false]{trace} code in particular.

% Reason 2
	\item The models are adequately complex. It is complex that $5$ flow regimes are involved in a single phenomena: multiple sub-models, parametrized with numerous inputs, with multivariate outputs (both time- and space-dependent).
	But as the source of data is from reflooding \glspl[hyper=false]{setf}, real plant system (and full scale) effects can be excluded and the ensuing analysis can be concentrated on limited set of models.
	In fact, as already pointed out, reflooding \glspl[hyper=false]{setf} were designed to validate and (to calibrate) reflood models in system codes.

% Reason 3
	\item Multiple data of various types (taken with different experimental conditions) are typically available from experiment within the same facility.
	As calibration in the present research is conducted using one experimental condition, it is important to validate the resulting calibration result against the data with different experimental condition albeit from the same experimental facility. 
	Moreover, additional data from another reflooding \glspl[hyper=false]{setf} are also available.
	This is important for future study of validating the proposed method further and of expanding it to calibration against data from multiple facilities. 

% Reason 4
	\item It is the model considered in the \gls[hyper=false]{premium} benchmark, thus there is possibility to compare the results of this research with the results of other participants of the benchmark\footnote{at least qualitatively due to different codes employed by different participants}.
	
\end{itemize}

As such, while it is important to acknowledge that reflood simulation and the associated relevant model (or models) are only parts of a large and complex thermal-hydraulics systems code,
they can provide a representative and relevant illustration on the particulars of analyzing and calibrating a thermal-hydraulics system code using experimental data from \gls[hyper=false]{setf} in general; providing a suitable testing ground for the proposed methods.

% Closing
As a final note, the thermal-hydraulics system code considered in this thesis is the \glsfirst[hyper=false]{trace} code developed by the \glsfirst[hyper=false]{usnrc}.
The main reason to consider solely this particular code in the present thesis is the fact that \gls[hyper=false]{trace} is the thermal-hydraulics system code used for the purpose of Swiss nuclear power plant safety analysis conducted withing the \glsfirst[hyper=false]{stars} program \cite{PSI2017} at the \glsfirst[hyper=false]{psi}.
\section{Overview of the Thesis}\label{sec:intro_overview}

This doctoral thesis is organized into seven chapters.
The $3$ statistical frameworks introduced in the previous section, 
bookended by a review of the concerned physical model and an independent validation study, 
becomes the central part of the thesis.

Following this introduction, 
the second chapter gives an overview of the system thermal-hydraulics code TRACE with an emphasis on its reflood phenomena modeling and simulation.
The chapter also introduces the reflood experiment at the \gls{feba} facility which serves as the experimental basis of this work
as well as its model in TRACE code.
Referring to the first part of the methodology in Fig~.\ref{fig:methodological_roadmap}, 
the chapter includes the selection of initial parameters relevant for reflood simulation and the propagation of their uncertainties on the code prediction.

Chapter~\ref{ch:gsa} provides the application of statistical sensitivity analysis method for the reflood model.

%******************************************************
\section{Chapter Summary}\label{sec:gp_chapter_summary}
%******************************************************

The functional approximation part of the proposed statistical framework has been presented in this chapter.
The goal of such an approximation was to evaluate the output of a computer simulation code for an arbitrary input much faster.
The approximation is based on Gaussian stochastic process resulting in a statistical metamodel.
As the dimensionality of the output is large, in the order of tens of thousands, a dimension reduction step is adopted by means of \gls[hyper=false]{pca} (an approach similar to what was adopted in Chapter 3).

The results obtained on the \gls[hyper=false]{trace} model \gls[hyper=false]{feba} is reasonable.
Though the prediction error can at times be large, the metamodel gives an overall good performance on average and in context for the three types of multivariate output (clad temperature, pressure drop, and liquid carryover). 
The limitation of the approach is mainly for the output which exhibits strong non-linearity and discontinuity (such as the quenching in the clad temperature transient).
This, in turn, is due to the use of \gls[hyper=false]{pca} as the (linear) dimension reduction tool.
As such, a first step of improvement in this regard can be aimed toward replacing \gls[hyper=false]{pca} with another, more advanced dimension reduction tool.

Using the \gls[hyper=false]{gp} \gls[hyper=false]{pc} metamodel as the surrogate for \gls[hyper=false]{trace} run, 
the prediction for arbitrary model parameters values can be made much faster ($< 5 [s]$ per metamodel evaluation vs. $6-15 \, [min]$ per \gls[hyper=false]{trace} run).
As such the metamodel constructed in this chapter can be used as the basis for Bayesian model calibration which requires tens if not hundreds of thousands function evaluations. 
However, it is also important to note that the time required for the construction of the metamodel and as well as for its convergence study has to be taken into account.
The training, validation, and testing data have to be generated from actual code runs.
Additionally, the model fitting step to estimate \gls[hyper=false]{gp} metamodel hyper-parameters is an optimization problem that can easily become expensive for large training samples of large dimension (large number of input parameters).
 
The study also confirms that the size of the training sample is the main factor in determining the predictive performance of the metamodel.
The choice of covariance function has some impact especially in relation to the stability of the performance,
while the choice of experimental design has a neglible impact on the performance.

%The resulting posterior uncertainty, derived from one set of experimental condition, is verified by propagating it on the other \gls[hyper=false]{feba} tests.
