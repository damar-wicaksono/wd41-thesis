%***************************************************************************************************************************************************
\chapter[Sensitivity Analysis]{Sensitivity Analysis: Understanding Model Input/Output Relationship Under Uncertainty}\label{ch:sensitivity_analysis}
%***************************************************************************************************************************************************

As mentioned in the introduction, describing and understanding properly the impact of model parameters variations on the model prediction are an essential part of model development and assessment. 
This chapter deals with the application of global and statistical \gls{sa} to analyze the \gls{feba} model in \gls{trace} in order to investigate the effect of variation.

After first presenting the notational convention used in the present chapter in Section~\ref{sec:sa_statistical_framework},
the proposed methodology is presented.
The methodology leverages various developments in global \gls{sa} and \glsentryfull{fda} methods and follows three key underlying ideas.

The first idea, presented in Section~\ref{sec:sa_time_dependent_variation}, is to reduce the dimensionality of the output space while keeping the interpretability of the results by utilizing techniques derived from \gls{fda} \cite{Ramsay2005}.
Section~\ref{sec:sa_parameters_screening} introduces the second idea, which is to reduce the dimensionality of the input parameter space through screening analysis using the Morris method \cite{Morris1991,Campolongo2011}. 
The third and final idea is to investigate, quantitatively and in more detail, the effect of variation of parameters on the overall time-dependent output variation. 
This is done through variance-based SA using the Sobol’-Saltelli
method \cite{Sobol2001, Saltelli2002}, which is presented in Section~\ref{sec:sa_variance_decomposition}.

The methods are then applied to analyze the \gls{feba} model in \gls{trace} to understand better its inputs/outputs relationship under the imposed uncertainty on its input parameters.
The results are presented and discussed in Section~\ref{sec:sa_application_to_feba}.
Finally, Section~\ref{sec:sa_chapter_summary} closes the chapter with a summary.

\section{Statistical Framework}\label{sec:gp_statistical_framework}
\section{Describing Variation of Time-Dependent Output}\label{sec:sa_time_dependent_variation}

Ramsay and Silverman \cite{Ramsay2005} popularized \gls{fda}, which refers to statistical analyses of data that are functions.
The main assumption of \gls{fda}, as opposed to a more conventional multivariate analysis, is that data present sufficient smoothness, defined by existence of derivatives up to a certain order.
Another distinguishing feature of \gls{fda}, as opposed to time-series analysis or spatial statistics, is the availability of numerous replications of such data (i.e., set of functions) produced by the same or similar underlying process. 
The goal of \gls{fda} related to this work is to describe the overall variation of a set of functions using a smaller set of scalars.
These scalars, in turn, can be used as the \gls{qoi} for \gls{sa}.

\subsection{Functional Output Representation}\label{sub:sa_spline}

\subsection{Curve Registration by Landmarks}\label{sub:sa_registration}

\subsection{Functional Principal Component Analysis}\label{sub:sa_fpca}

Separation of phase variation from magnitude variation by registration procedure allows for the definition of a proper mean function.
With respect to that, the notion of functional variation can be defined.
The covariance function of a set of function realizations $\{y_n(t);n = 1, 2, \cdots, N; t \in [t_a,t_b]\}$ from a random process $Y$ is defined as
\begin{equation}
	\nu (t_1, t_2) \equiv \frac{1}{N} \sum_{n=1}^{N} (y_n(t_1) - \bar{y}(t_1)) \cdot (y_n(t_2) - \bar{y}(t_2))
\label{eq:covariance_function}
\end{equation}

To extract more meaningful information from the covariance function, the function is often projected onto lower-dimensional space using an orthogonal decomposition.
This projection can be done through the functional principal component (fPC) analysis (fPCA) (also know as the Karhunen-Lo\'eve transform):
\begin{equation}
	\nu (t_1, t_2) = \sum_{j=1}^{+\infty} \rho_j \cdot \xi_j(t_1) \cdot \xi_j(t_2)
\label{eq:kl_transform}
\end{equation}
where $\rho_j$ is a series of ordered eigenvalues of decreasing values; 
$\xi_j(t)$ is the corresponding series of orthogonal eigenfunctions (or the fPC).
\section{Parameters Screening}\label{sec:sa_parameters_screening}

Screening methods are used to rank the importance of the model parameters using a relatively small number of model evaluations \cite{Saltelli2004}.
However, they tend to simply give qualitative measures.
That is, meaningful information resides in the rank itself but not in the exact importance of the parameters with respect to the output. 
These methods are particularly valuable in the early phase of a SA to identify the non-influential parameters of a model,
which then could be safely excluded from further detailed analysis. 
This exclusion step is important to reduce the size of the problem especially if a more expensive method is to be applied at the subsequent step. 
In this work, attention was paid to a particular screening method proposed by Morris \cite{Morris1991} with an extension proposed by Campolongo et al. \cite{Campolongo2011}

\subsection{Elementary Effects and One-at-a-Time Design}

Consider a model with $D$ parameters, where $\mathbf{x} = (x_1, x_2, \dots,x_D)$ is a vector of parameter value evaluated at point $\mathbf{x}$.
The elementary effect of the $d$-th parameter is defined as
\marginpar{elementary effect}
\begin{equation}
EE_d = \frac{y(x_1, \dots, x_d+\Delta,\dots,x_D) - y(x_1, \dots, x_d,\dots,x_D)}{\Delta}
\end{equation}
where $\Delta$, the grid jump, is chosen such that $\mathbf{x} + \Delta$ is still in the specified domain of the parameter space; $\Delta$ is a value in $\{\frac{1}{p-1}, \dots, 1 - \frac{1}{p-1}\}$, 
where $p$ is the number of (discretization) levels that partition the model parameter space into a uniform grid of points at which the model can be evaluated. 
For a given $p$, the grid constructs a finite distribution of $p^{D-1}[p - \Delta(p-1)]$ elementary effects for each input parameter.

The elementary effect distributions for each of the parameters, evaluated across discretized input parameter space, 
\marginpar{\gls{oat} experimental design}
provide useful information on the importance of a parameter on the output.
Unfortunately, an exhaustive evaluation of all elementary effects for a given discretization levels suffers from a curse of dimensionality especially for numerous parameters and/or for reasonably fine discretization level\footnote{for $p = 8$ and $D = 20$ the total number of evaluations for exhaustive computation for $EE$ is $\approx 6 \times 10^17$ for each parameter}.
Consequently, a class of design of experiment that only change one parameter at a time (\gls{oat}) are devised to estimate the statistics of the distributions.
  
The key idea behind the original Morris method is in initiating the model evaluations from various \textit{nominal} points, $\mathbf{x}$,
\marginpar{trajectory OAT design}
randomly selected over the grid and then gradually advancing one grid jump, perturbing one parameter at a time.
The order of perturbation (i.e., which dimension to perturb first) and the direction of the perturbation (i.e., whether it is added or subtracted) are also randomized between replication to replication.
As such, different replication generates different starting nominal point as well as different order and sign (but with with the same size) of perturbation.
The \gls{oat} computer experimental design complemented with this requirement is known as the trajectory design \cite{Ruano2012}.
Fig.~\ref{fig:illustrate_oat_design} (left) illustrates a trajectory design in two-dimensional input parameter space discretized in $6$ levels with 4 design replications.
\begin{figure}[bth]
	\centering
	\includegraphics[width=1.0\textwidth]{../figures/illustrateOATDesign/illustrateOATDesign.png}
	\caption[Illustration of One-at-a-Time (OAT) design using trajectory and radial schemes]{Illustration of One-at-a-Time (OAT) design constructed using trajectory scheme (left) and radial scheme (right) each with 4 replications. The trajectory design is discretized in 6 levels, while the number of levels is irrelevant for radial design. Filled circles are the base/nominal points and crosses are the perturbed levels.}
	\label{fig:illustrate_oat_design}
\end{figure}

To remove the requirement to specify a method-specific parameter $p$ (the number of levels),
\marginpar{radial OAT design}
Campolongo et al.\cite{Campolongo2011} proposed to use a radial scheme coupled with Sobol' quasirandom sequence.
In a single replication of this particular OAT design, 
each parameter is perturbed relative to a \textit{base/nominal} point 
which is not required to be located in a predetermined grid.
The size and sign of the perturbation is also allowed to vary from parameter to parameter in different replication.
As such, radial design implicitly incorporates several additional possible sources of variation in the method 
that can potentially bias the estimation of elementary effects.
Because the size of parameter perturbations varies, the definition of the elementary effects is slightly changed to,
\marginpar{elementary effect for radial design}
\begin{equation}
EE_d = \frac{y(x_1, \dots, x_d+\Delta x_d,\dots,x_D) - y(x_1, \dots, x_d,\dots,x_D)}{\Delta x_d}
\end{equation}
where now each parameter at each design replication has its corresponding perturbation size $\Delta x_d \in [-1,1]$ such that $x_d + \Delta x_d \in [0,1]$.

An illustration of a radial design in two-dimensional input space with $4$ base points is shown in Fig.~\ref{fig:illustrate_oat_design} (right).
 
\subsection{Statistics of Elementary Effects and Sensitivity Measures}

Consider now that an $N_R$ number of elementary effects associated with the $d$-th parameter have been sampled from the finite distribution of $EE_d$.
The statistical summary of sampled $EE_d$ based on a given number of OAT design replications generated using either the trajectory or the radial design can be calculated.
The first is the arithmetic mean defined as,
\marginpar{the mean of the (sampled) elementary effects}
\begin{equation}
	\mu_d = \frac{1}{N_R} \sum_{r = 1}^{R} EE^r_d
	\label{eq:sa_morris_mu}
\end{equation} 
where $EE^r_d$ is the elementary effect of the $d$-th parameter of the $r$-th sampled block.
The mean gives the global influence of the $d$-th parameter on the chosen output $y$.

The second statistical summary of interest is the standard deviation of the (sampled) elementary effects associated with the $d$-th parameter from all the replications,
\marginpar{the standard deviation of the (sampled) elementary effects}
\begin{equation}
	\mu_d = \sqrt{\frac{1}{N_R} \sum_{r = 1}^{R} (EE^r_d - \mu_d)^2}
	\label{eq:sa_morris_sd}
\end{equation} 
The standard deviation gives an indication of the present of nonlinearity and/or interactions between the $d$-th parameter and other parameters.

As a change in a parameter value might have a changing sign on the change of the output,
a cancellation effect on the estimation of $mu_d$ might occur 
(as can be the case for a non-monotonic function).
Campolongo et al. \cite{Campolongo2011} proposed to take the absolute value of the sampled elementary effects before taking the mean to circumvent this issue.
It is defined as,
\marginpar{the mean of the (sampled) absolute elementary effects}
\begin{equation}
	\mu^*_d = \frac{1}{N_R} \sum_{r = 1}^{R} |EE^r_d|
	\label{eq:sa_morris_mustar}
\end{equation}
Note that although the overall sign of output perturbation is lost by using this measure,
its use is justified if parameters are to be ranked based on a single importance measure.

The three aforementioned statistical summaries, when evaluated over a large number of replications $N_R$,
can provide global sensitivity measures of the importance of the $d$-th parameter.
As indicated by Morris \cite{Morris1991}, there are three possible categories of parameter importance based on those statistics:
\begin{enumerate}
	\item Parameters with noninfluential effects, i.e., the parameters that have relatively small values of both $\mu_d$ (or $\mu^*_d$) and $\sigma_d$.
	The values of both indicate that parameter has a negligible overall effect on the model output.
	\item Parameters with linear and/or additive effects, i.e., the parameters that have relatively large value of $\mu_d$ (or $\mu^*_d$) and relatively small value of $\sigma_d$.
	The small value of $\sigma_d$ and the large value of $\mu_d$ (or $\mu^*_d$) indicate that the variation of elementary effects across replications is small while the magnitude of the effect itself is consistently large for the perturbations in the parameter space.
	\item Parameters with nonlinear and/or interaction effects, i.e., the parameters that have a relatively small value of $\mu_d$ (or $\mu^*_d$) and a relatively large value of $\sigma_d$.
	Opposite to the previous case, a small value of $\mu_d$ (or $\mu^*_d$) indicates that the aggregate effect of perturbation is relatively small (or in the case of $\mu$, can be close to zero) while a large value of $\sigma_d$ indicates that the variation of the effect is large; the effect can be large or negligibly small depending on the values of the other parameters at which the model is evaluated.
	Such large variation is a symptom of nonlinear effects and/or parameter interaction.
\end{enumerate}

Such classification makes parameter importance ranking and, in turn, screening of noninfluential parameters possible.
However, the procedure is done rather qualitatively, and this is illustrated in Fig.~\ref{fig:illustration_morris_result}, 
which depicts a typical parameter classification derived from visual inspection of the elementary effects statistics on the $\sigma$ vs. $\mu^*$ plane.
The notions of influential and noninfluential are based on the relative locations of those statistics in the plane.
Typically, the noninfluential ones are clustered closer to the origin (relative to the more influential ones) with a pronounced boundary such as depicted in Fig.~\ref{fig:illustration_morris_result}. 
Admittedly, if these statistics are spread somewhat uniformly across the plane, 
the distinction would be more ambiguous and problematic\footnote{In such a case, a more advanced classification such as the ones based on clustering techniques might be helpful to identify a finer structure of the parameters importance}.
Furthermore, for a parameter with a large valued of both $\mu^*$ and $\sigma$,
the method cannot distinguish between nonlinearity effect from parameter interaction effect on the output.

\begin{figure}[bth]
	\centering
	\includegraphics[scale=0.90]{../figures/illustrateMorrisResult/illustrateMorrisResult.png}
	\caption[Illustration of a typical parameter importance classification based on Morris screening method]{Illustration of a typical parameter importance classifications obtained from the Morris screening method. The importance of each parameter is defined relative to each other with respect to their location in the $\sigma - \mu_*$ plane. Each dot represents a parameter, and the line corresponds to twice the standard error of the mean (SEM) indicating the relative magnitude of the standard deviation to the mean.}\label{fig:illustration_morris_result}
\end{figure}

%********************************************************************
\section{Variance Decomposition}\label{sec:sa_variance_decomposition}
%********************************************************************

Variance-based methods for global sensitivity analysis use variance as the basis to define a measure of input parameter influence on the overall output variation \cite{Cacuci2004}.
In a statistical framework of sensitivity and uncertainty analysis, 
this choice is natural because variance (or standard deviation) is often used as a measure of dispersion or variability in the model prediction \cite{Saltelli2008}.
The  dispersion, in turn, can measure the level precision of the prediction due when the input parameters are considered uncertain.

This section first presents a method to decompose the model output variance into the contributions from the individual variances of the inputs.
Then, two sensitivity measures based on the decomposition are introdiced a method for their estimation is presented.

%--------------------------------------------------------------------
\subsection{High-Dimensional Model Representation}\label{sub:sa_hdmr}
%--------------------------------------------------------------------

Consider once more a mathematical model $f: \bm{x} \in [0,1]^D \mapsto y = f(\bm{x}) \in \mathbb{R}$.
The high-dimensional model representation (HDMR) of $f(\bm{x})$ is a linear combination of functions with increasing dimensionality up to the dimension of $\bm{x}$ \cite{Li2001},
\begin{equation}
	\begin{split}
		f(\bm{x}) = f_o & + \sum_{d_1 = 1}^{D} f_d(x_d) + \sum_{1 \leq d_1 < d_2 \leq D} f_{d_1,d_2} (x_{d_2}, x_{d_1}) + \cdots  \\
	                      & + f_{1,2,\cdots,D} (x_1, x_2, \cdots, x_D)
	\end{split}
\label{eq:sa_hdmr}
\end{equation}
where $f_o$ is a constant.
The representation in Eq.~(\ref{eq:sa_hdmr}) is unique if the following condition \cite{Sobol2001}
\begin{equation}
    \int_{0}^{1} f_{d_1, d_2, \cdots d_i}(x_{d_1}, x_{d_2}, \cdots, x_{d_i}) d_{x_{d_m}} = 0 \,;
		\, \text{for}\quad m = 1,2,\cdots,s
\label{eq:sa_unicity}
\end{equation}
is verified for all $i \in {1, \cdots, D}$ and 
any corresponding ordered combination of dimensions $1 \leq d_1 < d_2 < \cdots < d_i \leq D$ of the input parameter space.

Assume now that $\bm{\mathcal{X}}$ is a random vector of independent and uniform random variables over a unit hypercube
$\{\Omega = \mathbf{x} \, | \, 0 \leq x_i  \leq 1; i = 1,\cdots, D\}$ such that
\begin{equation}
	\mathcal{Y} = f(\bm{\mathcal{X}})
\label{eq:sa_random_function}
\end{equation}
where $\mathcal{Y}$ is a random variable, resulting from the transformation of the random vector $\mathbf{X}$ by the function $f$.
Using Eq.~(\ref{eq:sa_unicity}) to express each term in Eq.~(\ref{eq:sa_hdmr}), it follows that
\begin{equation}
	\begin{split}
		f_o & = \mathbb{E}[\mathcal{Y}] \\
	  f_{d_1}(x_{d_1}) & = \mathbb{E}_{\sim d_1}[\mathcal{Y}|\mathcal{X}_{d_1}] - \mathbb{E}[\mathcal{Y}]\\
    f_{d_1,d_2}(x_{d_1},x_{d_2}) & = \mathbb{E}_{\sim d_1,d_2} [\mathcal{Y}|\mathcal{X}_{d_1}, \mathcal{X}_{d_2}] \\
																 & - \mathbb{E}_{\sim d_1}[\mathcal{Y}|\mathcal{X}_{d_1}] - \mathbb{E}_{\sim e}[\mathcal{Y}|\mathcal{X}_{d_2}] - \mathbb{E}[\mathcal{Y}] 
	\end{split}
\label{eq:sa_conditional_expectation}
\end{equation}

The same follows for higher-order terms in the decomposition. 
In Eq.~(\ref{eq:sa_conditional_expectation}),
$\mathbb{E}_{\sim \circ} [Y|\circ]$ is a conditional expectation operator,
where subscript symbol $\sim\circ$ in the subscript means that integration on the parameter space is carried out over all parameters except the one(s) in the subscript.
For instance, $\mathbb{E}_{\sim 1} [\mathcal{Y}|\mathcal{X}_1]$ refers to the conditional mean of $\mathcal{Y}$ given $\mathcal{X}_1$, and the integration is carried out for all possible values of parameters in $\mathbf{x}$ except $x_1$.
Note that because $\mathcal{X}_1$ is a random variable, the expectation conditioned on it is also a random variable.

Assuming that $f$ is square integrable, applying the variance operator on $\mathcal{Y}$ results in
\begin{equation}
	\begin{split}
		\mathbb{V}[\mathcal{Y}] = \sum_{d_1=1}^{D} \mathbb{V}[f_{d_1} (x_{d_1})] & + \sum_{1 \leq d_1 < d_2 \leq D} \mathbb{V} [f_{d_1,d_2} (x_d_1, x_d_2)] + \cdots \\
	                                                       & + \mathbb{V} [f_{1,2,\cdots,D} (x_1, x_2, \cdots, x_D)]
		\end{split}
\label{eq:sa_variance_decomposition}
\end{equation}

%------------------------------------------------------------------
\subsection{Sobol' Sensitivity Indices}\label{sub:sa_sobol_indices}
%------------------------------------------------------------------

Division by $\mathbb{V}[Y]$ aptly normalizes Eq.~(\ref{eq:sa_variance_decomposition}):
\begin{equation}
  1 = \sum_{d = 1}^{D} S_d + \sum_{1 \leq d < e \leq D} S_{d,e} + \cdots + S_{1,2,\cdots,D}
\label{eq:sa_normalized_variance}
\end{equation}

The Sobol' main-effect sensitivity index $S_d$ is defined as,
\marginpar{the main-effect index}
\begin{equation}
  S_d = \frac{\mathbb{V}_d [\mathbb{E}_{\sim d} [Y|X_d]]}{\mathbb{V}[Y]}
\label{eq:sa_main_effect_index}
\end{equation}
The numerator is the variance of the conditional expectation,
and the index is a global sensitivity measure interpreted as the amount of variance reduction in the model output if the parameter $X_d$ is fixed (i.e., its variance is reduced to zero).
The main-effect sensitivity index is also known in the literature as the \emph{first-order} sensitivity index 
as it captures only the effect of a single parameter variation on the model output without considering any interaction with other parameters \cite{Saltelli2002}.

A closely related sensitivity index proposed by Homma and Saltelli \cite{Homma1996} is the Sobol' total-effect index defined as,
\marginpar{the total-effect index}
\begin{equation}
  \begin{split}
    ST_{d} & = \frac{\mathbb{E}_{\sim d}[\mathbb{V}_{d}[Y|\mathbf{X}_{\sim d}]]}{\mathbb{V}[Y]}
           & = \frac{\mathbb{V}[Y] - \mathbb{V}_{\sim d}\left[\mathbb{E}_{d}\left[Y|\mathbf{X}_{\sim d}\right]\right]}{\mathbb{V}[Y]} \\
           & = 1 - \frac{\mathbb{V}_{\sim d}[\mathbb{E}_{d}[Y|\mathbf{X}_{\sim d}]}{\mathbb{V}[Y]}
  \end{split}
\label{eq:sa_total_effect_index}
\end{equation}
The index, also a global sensitivity measure, can be interpreted as the amount of variance left in the output if the values of all input parameters, 
\emph{except} $x_d$, can be fixed.

These two sensitivity measures can be related to the objectives of global SA for model assessment as proposed by Saltelli et al \cite{Saltelli2004,Saltelli2008}.
\marginpar{parameter prioritization objective}
The main-effect index is relevant to parameter prioritization in the context of identifying the most influential parameter 
since fixing a parameter with the highest index value would, \emph{on average}, lead to 
the greatest reduction in the output variation.

The total-effect index, on the other hand, is relevant to parameter fixing (or screening) in the context of identifying the least influential set of parameters since fixing any parameter that has a very small 
\marginpar{parameter screening objective}
total-effect index value would not lead to significant reduction in the output variation.
The use of the total-effect index to identify which parameter can be fixed or excluded is similar to that of the elementary effect statistics of the Morris method, 
albeit more exact but also more expensive to compute.
And finally, the difference between the two indices of a given parameter (Eqs.~(\ref{eq:sa_total_effect_index}) and (\ref{eq:sa_main_effect_index})) is used to quantify the amount of all interactions involving that parameters in the model output.

\section{Implementation}\label{sec:sa_implementation}

\subsection{The Morris Method}

\subsection{The Sobol'-Saltelli Method}

In principle, the estimation of the Sobol' indices defined by Eqs.~(\ref{eq:sa_main_effect_index}) and (\ref{eq:sa_total_effect_index}) can be directly carried out using \gls{mc} simulation.
\marginpar{brute force \\ Monte Carlo}
The most straightforward, though rather naive, implementation of \gls{mc} simulation to conduct the estimation is using two nested loops for the computation of the conditional variance and expectation appeared in both equations.

In the estimation of the main-effect index of parameter $x_d$, for instance, the outer loop samples values of $X_d$ while the inner loop samples values of $\mathbf{X}_{\sim d}$ (anything else other than $x_d$). T
The samples, in turn, are used to evaluate the model output.
In the inner loop, the mean of the model output (for a given value of $X_d$ but over many values of $\mathbf{X}_{\sim d}$) is taken. 
Afterwards, in the outer loop, the variance of the model output (over many values of $X_d$) is taken.
This approach can easily become prohibitively expensive as the nested structure requires two $N^2$ model evaluations per input dimension for either the main-effect and total-effect indices, while $N$ (the size of \gls{mc} samples) are typically in the range of $10^2 - 10^4$ for a reliable estimate. 
   
Sobol' \cite{Sobol2001} and Saltelli \cite{Saltelli2002} proposed an alternative approach that circumvent the nested structure of \gls{mc} simulation to estimate the indices.
The formulation starts by expressing the the expectation and variance operators in their integral form.
As the following formulation is defined on a unit hypercube of $D$-dimension parameter space where each parameter is a uniform and independent random variable,
explicit writing of the distribution within the integration as well as the integration range are excluded for conciseness.

First, the variance operator shown in the numerator of Eq.~(\ref{eq:sa_main_effect_index}) is written as
\begin{equation}
  \begin{split}
    \mathbb{V}_{d}[\mathbb{E}_{\sim d}[Y|X_d]] & = \mathbb{E}_{d}[\mathbb{E}_{\sim d}^{2}[Y|X_d]] - \left(\mathbb{E}_{d}[\mathbb{E}_{\sim d}[Y|X_d]]\right)^2 \\ 
                                               & = \int \mathbb{E}_{\sim d}^{2}[Y|X_d] dx_d - \left(\int \mathbb{E}_{\sim d}[Y|X_d] dx_d\right)^2
  \end{split}
\label{eq:ss_variance_integral}
\end{equation}
The notation $\mathbb{E}_{\sim \circ}[\circ | \circ]$ was already explained in Section~\ref{sub:sa_hdmr}, 
while $\mathbb{E}_{\circ} [\circ]$ corresponds to the marginal expectation operator 
where the integration is carried out over the range of parameters specified in the subscript. 

Next, consider the term conditional expectation shown in Eq.~(\ref{eq:ss_variance_integral}), which per definition reads
\begin{equation}
  \mathbb{E}_{\sim d} [Y|X_d] = \int f(\mathbf{x}_{\sim d}, x_d) d\mathbf{x}_{\sim d}
\label{eq:ss_expectation_integral}
\end{equation}
Note that $\mathbf{x} = \{\mathbf{x}_{\sim d}, x_d\}$.

Following the first term of Eq.~(\ref{eq:ss_variance_integral}), by squaring Eq.~(\ref{eq:ss_expectation_integral})
and by defining a dummy vector variable $\mathbf{x}^{\prime}_{\sim d}$, 
the product of the two integrals can be written in terms of a single multiple integrals
\begin{equation}
  \begin{split}
    \mathbb{E}_{\sim d}^{2} [Y|X_d] & = \int f(\mathbf{x}_{\sim d}, x_d) d\mathbf{x}_{\sim d} \cdot \int f(\mathbf{x}_{\sim d}, x_d) d\mathbf{x}_{\sim d} \\
                                    & = \int \int f(\mathbf{x}^{\prime}_{\sim d}, x_d) f(\mathbf{x}_{\sim d}, x_d) d\mathbf{x}^{\prime}_{\sim d} d\mathbf{x}_{\sim d}
  \end{split}
\label{eq:ss_multiple_integrals}
\end{equation}

Returning to the full definition of variance of conditional expectation in Eq.~(\ref{eq:ss_variance_integral}),
\begin{equation}
  \begin{split}
    \mathbb{V}_{d}[\mathbb{E}_{\sim d}[Y|X_d]] & = \int \int f(\mathbf{x}^{\prime}_{\sim d}, x_d) f(\mathbf{x}_{\sim d}, x_d) d\mathbf{x}^{\prime}_{\sim d} d\mathbf{x}_{\sim d} \\
                                               & \quad - \left(\int f(\mathbf{x}) d\mathbf{x}\right)^2
  \end{split}
\label{eq:ss_variance_integral_single}
\end{equation}

Finally, the main-effect sensitivity index can be written as an integral as follows:
\begin{equation}
  \begin{split}
    S_d & = \frac{\mathbb{V}_d [\mathbb{E}_{\sim d} [Y|X_d]]}{\mathbb{V}[Y]} \\
        & = \frac{\int \int f(\mathbf{x}^{\prime}_{\sim d}, x_d) f(\mathbf{x}_{\sim d}, x_d) d\mathbf{x}^{\prime}_{\sim d} d\mathbf{x}_{\sim d} - \left(\int f(\mathbf{x}) d\mathbf{x}\right)^2}{\int f(\mathbf{x})^2 d\mathbf{x} - \left( \int f(\mathbf{x}) d\mathbf{x}\right)^2}
  \end{split}
\label{eq:ss_main_effect_integral}
\end{equation}
The integral form given above dispenses with the nested structure of multiple integrals in the original definition of main-effect index.
It is the basis of estimating sensitivity index using \gls{mc} simulation in this thesis, hereinafter referred to as the Sobol'-Saltelli method.
The same procedure applies to derive the total effect-index.

For $N$ number of \gls{mc} samples and $D$ number of model parameters, the \gls{mc} simulation procedure to estimate the sensitivity indices follows the sampling and resampling approach adopted in~\cite{Sobol2001,Saltelli2002,Homma1996}.

First, generate two $N \times D$ independent random samples from a uniform independent distribution in $D$-dimension, $[0,1]^D$:
\begin{equation}
A = 
\begin{pmatrix}
a_{11}  & \cdots  & a_{1D}\\
\vdots	& \ddots & \vdots\\
a_{N1}  & \cdots  & a_{ND}\\
\end{pmatrix}
;\quad B = 
\begin{pmatrix}
b_{11}  & \cdots  & b_{1D}\\
\vdots	& \ddots & \vdots\\
b_{N1}  & \cdots  & b_{ND}\\
\end{pmatrix}
\label{eq:ss_two_samples}
\end{equation}

Second, construct $D$ additional design of experiment matrices where each matrix is matrix $A$ with the $d$-th column substituted by the $d$-th column of matrix $B$:\begin{equation}
  \begin{split}
  & A_{B}^1 = 
  \begin{pmatrix}
    b_{11}  & \cdots  & a_{1D}\\
    \vdots	& \ddots & \vdots\\
    b_{N1}  & \cdots  & a_{ND}\\
  \end{pmatrix} \\
  & A_{B}^{d} = 
  \begin{pmatrix}
    a_{11}  & \cdots & b_{1d} & \cdots & a_{1D}\\
    \vdots	& \cdots & \vdots & \cdots & \vdots\\
    a_{N1}  & \cdots & b_{Nd} & \cdots & a_{ND}\\
  \end{pmatrix} \\
  & A_{B}^{D} = 
  \begin{pmatrix}
    a_{11}  & \cdots  & b_{1D}\\
    \vdots	& \ddots & \vdots\\
    a_{N1}  & \cdots  & b_{ND}\\
  \end{pmatrix}
  \end{split}
\label{eq:ss_two_samples}
\end{equation}

Third, rescale each element in the matrices of samples to the actual values of model parameters according to their actual range of variation through iso-probabilistic transformation.

Fourth, evaluate the model multiple times using input vectors that correspond to each row of $A$, $B$, and all the $A_B^d$.

Fifth, extract the \gls{qoi}s from all the outputs and recast them as vectors.
The main-effect and total-effect indices are then estimated using the estimators described below.

\begin{table}[h]
	\myfloatalign
	\caption[Monte Carlo estimators to estimate the main-effect indices]{Two \gls{mc} estimators for the terms in Eq.~(\ref{eq:ss_main_effect_integral}) to estimate the main-effect indices (the sum is implicitly over all samples $N$)}
	\label{tab:ss_main_effect_estimator}
	\begin{tabularx}{\textwidth}{cll} \toprule
		\tableheadline{Estimator}     & \tableheadline{$\mathbb{E}^2[Y] = \left( \int f d\mathbf{x}\right)^2$} & \tableheadline{$\mathbb{V}[Y] = \int f^2 d\mathbf{x} - \left( \int f d\mathbf{x}\right)^2$} \\ \midrule 
		Saltelli             & $\frac{1}{N} \sum f(A)_n \cdot f(B)_n$  & $\frac{1}{N}\sum f(A)_n^2 - \left(\frac{1}{N}\sum f(A)_n\right)^2$ \\
    \cite{Saltelli2002}  &                                         & \\
		Janon et al.         & $\left(\frac{1}{N} \sum \frac{f(B)_n + f(A_B^d)_n}{2}\right)^2$  & $\frac{1}{N} \sum \frac{f(B)_n^2 + f(A_B^d)_n^2}{2}$ \\
    \cite{Janon2014}     &                                                                  & $-\left(\frac{1}{N} \sum \frac{f(B)_n^2 + f(A_B^d)_n^2}{2}\right)^2$ \\
		\bottomrule
	\end{tabularx}
\end{table}

The computational cost associated with the estimation of all the main-effect and total-effect indices is $N \times (K + 2)$ code runs,
\marginpar{computational cost: \\ Morris vs. Sobol'-Saltelli}
where $N$ is the number of \gls{mc} samples and $K$ is the number of parameters.
As a comparison, the cost for Morris method to compute the statistics of elementary effect is $N_R \times (K + 1)$ code runs,
where $N_R$ is the number of OAT design replications.
In either methods, the number of samples $N$ (in the case of the Sobol'-Saltelli method) and replications $N_R$ (in the case of the Morris method)
determines the precision of the estimates.
A larger number of samples (and replications) increases the precision.
Note, however, that in practice the typical number of Morris replications is between $10^1 - 10^2$~\cite{Saltelli2010}, 
while the number of \gls{mc} samples for the Sobol' indices estimation amounts to $10^2 - 10^4$~\cite{Sobol2001}.
%*************************************************************************************************************
\chapter[Introduction]{Quantifying Uncertainty of Computer Model: Forward and Backward}\label{ch:introduction}
%*************************************************************************************************************

Lorem Ipsum

%************************************************************************************************************
\section{Computer Simulation and Safety Analysis of Nuclear Power Plant}\label{sec:intro_computer_simulation}
%************************************************************************************************************

The ubiquity of computer simulation applications in many fields of science and engineering results in an even more pervasive definitions of the term \textit{scientific computer simulation} itself 
and other associated terms such as \textit{model} and \textit{simulation}.
\marginpar{scientific computer simulation}
Though most of the definitions in the literature are not necessarily in contradiction to each other, 
to avoid confusion, this thesis adopts a recent definition proposed by Kaizer et al.\cite{Kaizer2015} quoted below,

\begin{quote}
	Scientific Computer Simulation is the imitation of a behavior of a system, entity, phenomenon, or process in the physical universe 
	using limited mathematical concepts, symbols, and relations through the exercise or use of scientific computer model.
\end{quote}

There are three main points highlighted in this definition.
\marginpar{model, simulation, and computer simulation}
First, this definition accentuates the difference between a \emph{model} and its \emph{simulation}.
Specifically, the former deals with the notion of representation, while the latter deals with the notion of imitation of a behavior.
Secondly, what makes a model scientific is that it treats physical phenomena or the behavior of a real world system as its subject.
Thirdly and finally, the modifier \emph{computer} in the definition makes it explicit that digital computer is used to solve whatever mathematical models serve as the representation.
This is usually the case for mathematical models that cannot be solved analytically.
Though this limitation what makes a solution of the model possible in the first place, 
it also affects the solution and its possible interpretation and thus many computational-related aspects also need to be comprehensively considered.
\footnote{Beven \cite{Beven2009} articulates this distinction into three levels of model: perceptual model (i.e., theoretical description of some physical phenomena), formal model (i.e., the mathematical description of it), and procedural model  (i.e., computer implementation of the formal model). For many physical system modeling applications, only the procedural model is able to make a quantitative prediction of the system. These distinctions are useful in acknowledging the level of approximation involved in moving from perceptual to procedural model.}
\section{Computer Experiment}\label{sec:intro_computer_experiment}

Granted, a simplicistic model cannot be expected to imitate all the important features of a complex physical phenomenon.
And yet, there is a tendency of developing and applying overly complex model, with numerous parameters and multiple non-linear relationships, for computer simulation.
This tendency has invited many critics over the year (citation needed).
In nuclear science and engineering, for instance, Zuber \cite{Zuber2001} and Wullf \cite{Wulff2007} have long critized the development and the use of multi-fluid model for thermal-hydraulics simulation as high in complexity and in maintenance cost, but low in fidelity and its usefulness.

The goal of computer simulation, complex or otherwise, is to provide prediction.
The main characteristic (and source of criticism) of using multi-parameter complex model for simulation is that the the relationship between numerous inputs and outputs becomes increasingly opaque.
The impact of changing one parameter alone, or especially together, on the prediction is hard to disentangle or intuit.
Furthermore, as appropriate values of inputs might not be fully known, they are often given simply over range of interest containing different possible permissible values.
It is thus seldom the case that one single simulation is sufficient to provide a reliable answer according to hardly any objective of computer simulation.
As analytical solution
%*************************************************************************************************************************
\section{Uncertainty Quantification in Nuclear Engineering Thermal-Hydraulics}\label{sec:intro_uncertainty_quantification}
%*************************************************************************************************************************

% Introductory Paragraph
Before continuing the discussion of uncertainty analysis of code predictions, it will be worthwhile to define some additional terminologies to avoid later confusion.

In making a connection with the notion of \emph{simulator} introduced in Section~\ref{sec:intro_computer_simulation}, 
recall that from Fig.~\ref{fig:ch1_th_system_code} an \emph{input deck} is distinct from the code itself.
Fig.~\ref{fig:ch1_simulator_io} depicts the notion of simulator of a thermal-hydraulics system in a more generic way, as an input/output model.
\begin{figure}[bth]	
	\centering
	\includegraphics[width=\textwidth]{../figures/chapter1/figures/simulator_io}
	\caption[Simplified illustration of a simulator as an input/output model.]{Simplified illustration of a simulator as an input/output model.}
	\label{fig:ch1_simulator_io}
\end{figure}

Indeed input deck defines specific problem (i.e., system) of interest.
It includes the specifications for geometrical configuration (i.e., the nodalization), choice of material and fluid involves, as well as initial and boundary conditions.
It may also include the setting for the numerical solver.
Some of those specifications (such as the boundary conditions, etc.) are parametrized and constitutes \emph{controllable inputs} denoted by $\bm{x}_c$\footnote{later on, \emph{controllable} inputs corresponds to the equivalent parameters in a physical experiment which can be controlled by the experimentalist.}.
The conservation equations of the code are closed with additional set of closure laws (and other sub-models) $\mathcal{M}_i(\bm{x}_c, \bm{x}_m, \bm{u})$.
These closure laws are, in turn, parametrized with a set of model specific parameters denoted by $\bm{x}_m$ which is referred to as the \emph{physical model parameters}.
 
Specifying the input deck, as far as user is concerned, completely defines the problem and the code will solve the conservation equations which output the dynamic state of relevant physical variables $\mathbf{u}(\bm{r}, t)$ (e.g., fluid pressure, temperature, wall temperature, etc.).
In practice, these raw simulation outputs are further post-processed to obtain the so-called \glspl[hyper=false]{qoi} the are relevant to the problem at hand.

%--------------------------------------------------------------------------
\subsection{Forward Uncertainty Quantification}\label{sub:intro_uq_forward}
%--------------------------------------------------------------------------

% Best-estimate, limitation
As explained, best-estimate analysis uses more realistic modeling assumptions for analyzing transient behavior of \gls[hyper=false]{npp}.
It attempts as realistically as possible to describe the behavior of the relevant physical processes occur during the plant transient.
And yet, even the best available understanding of the physical process is still limited.
Understanding of complex phenomena might not yet adequate and data support for some processes can be very limited.s
Simplifying assumptions, approximations, and expert judgments to some degree are unavoidable and still required to have a complete analysis.

% Best-estimate, plus uncertainty
Hence, best-estimate analysis has to be complemented with uncertainty analysis.
\marginpar{Best-estimate plus uncertainty}
The ultimate goal of uncertainty analysis is to associate code prediction with its uncertainty.
These combined quantities are then compared with certain regulator safety limits (e.g., \gls[hyper=false]{pct}) to check whether the limits still fall outside the uncertainty band of the code prediction.

% Source of possible uncertainties
There are several known sources of uncertainty that render the prediction on $\bm{u}(\mathbf{r},t)$ and its derived quantities uncertain.
The following are the sources of primary interest in the present research:
\begin{enumerate}
	\item 1 
  \item 2
  \item 3
\end{enumerate}

% Forward uncertainty quantification, Inputs as random variables
In statistical uncertainty analysis, the controllable inputs and physical model parameters are modeled as random variables ($\bm{\mathcal{X}}_c$ and $\bm{\mathcal{X}}_m$, respectively) equipped with probability distributions.
\marginpar{Forward uncertainty quantification}
Then using \gls[hyper=false]{mc} technique, samples are generated from their respective distributions and they are used to run the code multiple times.
Finally, statistics of code outputs (raw or post-processed), are summarized to obtain the uncertainty measure of the prediction.
In other words, the uncertainties in the controllable inputs and physical model parameters are \emph{propagated forward} through the code to quantify the uncertainty of the predictions as shown in Fig.~\ref{fig:ch1_simulator_uq_forward}.
The practice of propagating uncertainty by \gls[hyper=false]{mc} is widely accepted in the nuclear engineering thermal-hydraulics community \cite{Lellouche1990,Glaeser1994,Glaeser2008}.
\begin{figure}[!bth]	
	\centering
	\includegraphics[width=\textwidth]{../figures/chapter1/figures/simulator_uq_forward}
	\caption[Simplified flowchart of forward uncertainty quantification of a simulator prediction.]{Simplified flowchart of forward uncertainty quantification of a simulator prediction. Notice that the simulator has been parametrized by the controllable inputs and physical model parameters, each of which are represented as random variable.}
	\label{fig:ch1_simulator_uq_forward}
\end{figure}

% Source of uncertainty, initial and boundary condition

% Source of uncertainty, physical model parameters

% Model parameters
The physical model parameters, however, are conceptually different.
\marginpar{Model parameters}
The physical models referred to in this thesis are usually represented either in the form of correlation or phenomenological (mechanistic) model.
The parameters associated with these models are derived from experimental data.
They can either represent physically meaningful quantities (e.g., reaction rate coefficient) or not (i.e., a tuning parameter).
In either way, there are uncertainties associated with these parameters especially as the conditions of their intended use is making prediction can be (at times, very) different than the conditions of their derivation.
The conceptual distinction related to model parameters will be revisited in Chapter~\ref{ch:bayesian_calibration}.

%--------------------------------------------------------------------------
\subsection{Inverse Uncertainty Quantification}\label{sub:intro_uq_inverse}
%--------------------------------------------------------------------------

% Introductory paragraph, motivation
The physical model parameters often cannot be measured directly.
In this situation, in order to estimate their values,
experiments with well-specified conditions are carried out and correlations (models) are fitted based on the experimental data. 
Recall that from the discussion of closure laws origin in Section~\ref{sub:intro_th_system_code},
this can also apply to phenomenological models where free parameters are allowed to be tuned according to match the experimental data.
Ultimately, the optimal value of the estimated parameter is implemented in the code.
As the code is used to simulate multiple phenomena during a transient, additional experiments are carried out in larger, more complex test facilities.
Finally, based on the data obtained, the models can be validated.

% Inverse uncertainty
To obtain the uncertainty associated with the model parameters obtained in the manner above, the problem can be posed as an inverse problem.
In this setting, given a set of experimental data $\{\mathbf{D}\}$ with a known controllable inputs $\mathbf{x}_c$, the task is then to infer the value of the physical model parameters.
It is important to acknowledge various sources of uncertainty previously mentioned:
experimental data and controllable inputs are observed but perhaps there remains residual uncertainty and observation error associated with them;
and the associated models are also only an approximation of the real physical processes with a possible, but unknown, systematic bias.
\bigfigure[pos=tbhp,
           opt={width=1.0\textwidth},
           label={fig:ch1_simulator_uq_inverse},
           shortcaption={Simplified flowchart of inverse uncertainty quantification of model parameters.}]
{../figures/chapter1/figures/simulator_uq_inverse}
{Simplified flowchart of inverse quantification for model parameters of a simulator.}

% Connection to PREMIUM Benchmark
The importance of characterizing the uncertainty in the physical models parameters was acknowledged by the \gls[hyper=false]{wgama} of the \gls[hyper=false]{oecd}/\gls[hyper=false]{nea}.
This led to the \gls[hyper=false]{premium} project.
Its main goal is to report the state-of-the-art of the available methodologies to quantify the uncertainty in the physical models parameters.
The following will briefly describe the project and highlight its main findings.

%-------------------------------------------------------------
\subsection{OECD/NEA PREMIUM project}\label{sub:intro_premium}
%-------------------------------------------------------------

% Introductory paragraph
The \gls[hyper=false]{premium} project was an activity launched by the \gls[hyper=false]{oecd}/\gls[hyper=false]{nea} in $2012$ and concluded in $2016$ with the aim to advance the methods for quantifying the uncertainties associated with the physical model parameters in \gls[hyper=false]{th} system codes.
It was the continuation of the previous project \gls[hyper=false]{bemuse}, which concetrated on the propagation and sensitivity analysis of the input uncertainties in large scale simulation (large break \gls[hyper=false]{loca}).
The main finding of \gls[hyper=false]{bemuse} can be found in \cite{Perez2011}.
The emphasis of the \gls[hyper=false]{premium} benchmark was placed on the derivation of the model parameters uncertainty and their validation.

% Scope of the Project

% Main Findings

\input{Chapters/1_introduction/premium}
\input{Chapters/1_introduction/bayesian_framework}
%*********************************************************************************
\section{Objectives and Scope of the Thesis}\label{sec:intro_objectives_and_scope}
%********************************************************************************* 

With a larger context provided above,
this section presents briefly and specifically the statement of the problem,
followed by the objectives as well as the scope of the present doctoral research.

%--------------------------------------------------------------------------
\subsection{Statement of the Problem}\label{sub:intro_statement_of_problem}
%--------------------------------------------------------------------------

% Introductory Paragraph
The development of closure laws for reflooding described in \cite{Nelson1992,USNRC2012} showed the difficulties and the amount of assumptions used.
In a nutshell, system code development is an effort to consolidate correlations and mechanistic models, to create a phenomenological-based simulation code that can provide best-estimate results.
This consolidated effort results in a code that can simulate wide range of transients foreseen in nuclear power plant operation in a best-estimate manner.
Alas, to come up with a consistent set of closure laws is a great challenge for code developers.

% Closure Laws Difficulty, Conceptual
The closure laws required to close the two-fluid model pose particularly difficult challenges \cite{Wulff2007}.
For instance, to have a correlation of heat transfer between the wall and the fluid, temperature data from each of the constituents are needed (i.e., the wall, the liquid phase, and the gas phase).
But measuring temperature of the individual phases in an arbitrary interfacial topology has its own technical difficulties to the extend that no such data exists or available to be implemented in the closure laws.
Additionally, the experiments to obtain hydrodynamic closure laws (e.g., interfacial friction factor, wall friction factor, etc.) were generally carried out in adiabatic conditions.
As a result, this excludes the coupling of any heat transfer phenomena between the phases and the wall in such correlation.

% Closure Laws Difficult, Practical
Furthermore, during the development of a simulation code, programming considerations also came into the picture.
For robustness, simplification is often required and continuity is enforced.
Transitionary flow regime between two known (observed) flow regimes for which experimental data is not available is modeled to be the average of the two bounding regimes.
Different code development, which used different assumptions and experimental database, comes up with different set of closure laws with their own parametrization (see for instance \cite{Nelson1992} for TRAC code and \cite{Bestion1990} for CATHARE code).
Several authors have expressed their concerns about the uncertainty stemming from the closure laws \cite{Wulff2007,Petruzzi2008a,DAuria2012}.

% an Illustration
As an example of the point given above, consider that in the \gls[hyper=false]{trace} code, after some derivations the interfacial drag coefficient closure law in the inverted slug flow regime $C_{i,\text{IS}}$ is given by,
\begin{equation*}
	C_{i,\text{IS}} = \hat{x}_{m,\text{SET}} \times \frac{1}{24} \frac{\rho_g}{\text{La}} \frac{(1-\alpha)}{\alpha^{1.8}} \,\,\,;\,\,\, \hat{x}_{m,\text{SET}} = 0.75 
\label{eq:intf_drag_isf}
\end{equation*}
where $\rho_g$ is the density of the gas phase;
$\text{La}$ is the Laplace number;
$\alpha$ is the void fraction;
and $\hat{x}_{m,\text{SET}}$ is a fitting parameter.

There are several remarks about the closure law given above.
First, the second term in the right-hand side was derived from experimental data but not directly.
In the inverted slug regime, saturated liquid core breaks up into ligaments.
These ligaments are \emph{assumed} to take form as prolate ellipsoid.
The drag coefficient of distorted droplet experimental database is then \emph{assumed}.
Then to take into account the multi-particle effect, the coefficient is divided by the void fraction $\alpha$ raised to the power of $1.8$ (this, in turn, was taken from experimental data of inertial regime).
Lastly, the first term of the equation, $\hat{x}_{m,\text{SET}} = 0.75$ was put \emph{to match}, \emph{to calibrate against} the experimental data from the FLECHT-SEASET reflood facility.
This first term, although clearly \emph{non-physical}, is an important tuning parameter of the model nevertheless.
Its uncertainty should be considered in uncertainty analysis, especially when reflood is expected to occur.
Yet, no statement regarding the associated uncertainty is given.
Several other such terms exist \cite{USNRC2012}. 

% Statement of Problem
As illustrated above, it is clear that models in thermal-hydraulics system code, to a certain extent, flawed.
Various experimental programs were carried out to gain better understanding of important phenomena,
and to validate (and, as noted above, to calibrate) the models.
Series of the experiments, carried out in \glspl[hyper=false]{setf} were aimed to reproduce limited part of the transient in a selected component following a postulated scenario.
For example, in the case of reflooding, several facilities existed and data were available (FEBA, PERICLES, etc.).
But, there has not been an orchestrated effort to incorporate the accumulated data into the calibration process of the physical models, in a systematic way, while acknowledging multiple sources of the uncertainty in the process.

%--------------------------------------------------
\subsection{Objectives}\label{sub:intro_objectives}
%--------------------------------------------------

% Introductory (Overall Objective)
The purpose of the doctoral research is to quantify the uncertainty of physical model parameters
implemented in a thermal-hydraulics system code.
The physical models of interest describe the phasic interactions in a complex multiphase flow during a reactor transient, namely heat, mass, and momentum exchanges between vapor, water and structures.
These models are parametrized by physical or empirical tuning parameters, the values of which are uncertain.
This results in uncertain code prediction of important safety quantities, such as the evolution of the fuel cladding temperature during a postulated reactor transient.

Adopting probabilistic framework to conform to the statistical uncertainty propagation widely
adopted in the field of nuclear engineering, the uncertainties in the parameters are represented in
form of probabilistic density functions or their approximation.
The derivation of these functions is posed as an inverse statistical problem following Bayesian framework as the parameters themselves are not directly observable.
The doctoral research thus aims to present a consistent set of strategies in deriving the uncertainty of such model parameters based on experimental data.
This is done by consolidating and adapting recent developments in the applied statistics literature:

% Aim 1 (Global sensitivity analysis)
\begin{enumerate}
	\item \emph{to analyze and to better understand} the inputs/outputs relationship in a computer simulation with uncertain inputs.
	This is aimed at answering the question whether the current physical model in thermal-hydraulics system code \gls[hyper=false]{trace} can be identified with the available experimental data from test facilities.
	In other words, how to select important parameters to be inferred.
	\Glsfirst[hyper=false]{gsa} methodologies can be used to assist in identifying which parameters can be calibrated using the available data.
	A test facility might have multiple types of data and although the information content might not be the same for the different types, it might be worthwhile to consider each one of them.
	Finally, for each of the different types,
	the analysis is also conducted on various derived \glspl[hyper=false]{qoi}, some of which explicitly consider the output as function.
	By doing so, it is hope that interesting model behavior with respect to its parameters perturbation can be revealed.

% Aim 2 (Statistical Metamodeling)
	\item \emph{to approximate} the inputs/outputs relationship in a complex computer simulation for a faster evaluation.
	The step is required as the statistical calibration method adopted in thesis is computationally expensive, requiring numerous code runs in the order of hundreds of thousands and beyond.
	This approximation is done through a \glsfirst[hyper=false]{gp} metamodel resulting in a statistical metamodel.
	The highly multivariate nature of the outputs (time- and space-dependent) is dealt by a dimension reduction technique.
	Build upon the results of previous step, only parameters that are identified to be influential are included in the construction of the metamodel.

% Aim 3 (Bayesian Calibration)
	\item \emph{to statistically calibrate} the physical model parameters against various relevant experimental data.
	The word \emph{to calibrate} carries a disparaging interpretation related to \emph{to tweak}.
	However, using a Bayesian statistical framework, the aim of calibration is extended to simultaneously quantify the uncertainty of the parameter estimation.
	The framework includes various sources of uncertainty which can be modeled using probabilistically, including the model bias term.
	At the end, the parameters of interest will be either in the form of distributions conditioned on the data or samples generated from such distributions which are useful in the uncertainty analysis of code prediction.

% Aim 4 (Extrapolation)
	\item \emph{to validate} the statistical calibration results against experimental data set not used in the calibration step.
	As calibration only conducted using experimental data of limited experimental conditions, it is important, at the minimum, to validate the proposed methods by demonstrating the applicability of the results to the simulation of the phenomena of the same facility in different experimental conditions. 

\end{enumerate}

%----------------------------------------
\subsection{Scope}\label{sub:intro_scope}
%----------------------------------------

% Introductory paragraph
Although the proposed set of strategies in this research can be applicable to the analysis and calibration of any system code physical model,
it is illustrated by its application on the models of particular importance during simulation of reflooding,
i.e., the so-called \gls[hyper=false]{postchf} flow regimes.
There are several reasons for this emphasis:
\begin{itemize}

	% Reason 1
	\item Reflooding is an important part in the simulation of \glspl[hyper=false]{npp} transient during \gls[hyper=false]{loca}.
	Modeling reflooding determines the appropriate representation of the dynamics of heat transfer phenomena during the effort to rewet an uncovered core.
	Of paramount interest is to estimate the time the rod can be expected to be rewet as well as the maximum temperature reached prior to rewet.
	Reflood is a transient with highly coupled hydrodynamic-heat-transfer effects and it challenges the assumption made on the implemented closure laws.
	Indeed several reflood experimental programs conducted in \glspl[hyper=false]{setf} existed and were designed to validate reflood models in system code.
	Unfortunately, no orchestrated effort was done so far to consolidate the generated data in general and into \gls[hyper=false]{trace} code in particular.

% Reason 2
	\item The models are adequately complex. It is complex that $5$ flow regimes are involved in a single phenomena: multiple sub-models, parametrized with numerous inputs, with multivariate outputs (both time- and space-dependent).
	But as the source of data is from reflooding \glspl[hyper=false]{setf}, real plant system (and full scale) effects can be excluded and the ensuing analysis can be concentrated on limited set of models.
	In fact, as already pointed out, reflooding \glspl[hyper=false]{setf} were designed to validate and (to calibrate) reflood models in system codes.

% Reason 3
	\item Multiple data of various types (taken with different experimental conditions) are typically available from experiment within the same facility.
	As calibration in the present research is conducted using one experimental condition, it is important to validate the resulting calibration result against the data with different experimental condition albeit from the same experimental facility. 
	Moreover, additional data from another reflooding \glspl[hyper=false]{setf} are also available.
	This is important for future study of validating the proposed method further and of expanding it to calibration against data from multiple facilities. 

% Reason 4
	\item It is the model considered in the \gls[hyper=false]{premium} benchmark, thus there is possibility to compare the results of this research with the results of other participants of the benchmark\footnote{at least qualitatively due to different codes employed by different participants}.
	
\end{itemize}

As such, while it is important to acknowledge that reflood simulation and the associated relevant model (or models) are only parts of a large and complex thermal-hydraulics systems code,
they can provide a representative and relevant illustration on the particulars of analyzing and calibrating a thermal-hydraulics system code using experimental data from \gls[hyper=false]{setf} in general; providing a suitable testing ground for the proposed methods.

% Closing
As a final note, the thermal-hydraulics system code considered in this thesis is the \glsfirst[hyper=false]{trace} code developed by the \glsfirst[hyper=false]{usnrc}.
The main reason to consider solely this particular code in the present thesis is the fact that \gls[hyper=false]{trace} is the thermal-hydraulics system code used for the purpose of Swiss nuclear power plant safety analysis conducted withing the \glsfirst[hyper=false]{stars} program \cite{PSI2017} at the \glsfirst[hyper=false]{psi}.
\section{Overview of the Thesis}\label{sec:intro_overview}

This doctoral thesis is organized into seven chapters.
The $3$ statistical frameworks introduced in the previous section, 
bookended by a review of the concerned physical model and an independent validation study, 
becomes the central part of the thesis.

Following this introduction, 
the second chapter gives an overview of the system thermal-hydraulics code TRACE with an emphasis on its reflood phenomena modeling and simulation.
The chapter also introduces the reflood experiment at the \gls{feba} facility which serves as the experimental basis of this work
as well as its model in TRACE code.
Referring to the first part of the methodology in Fig~.\ref{fig:methodological_roadmap}, 
the chapter includes the selection of initial parameters relevant for reflood simulation and the propagation of their uncertainties on the code prediction.

Chapter~\ref{ch:gsa} provides the application of statistical sensitivity analysis method for the reflood model.

%******************************************************
\section{Chapter Summary}\label{sec:gp_chapter_summary}
%******************************************************

The functional approximation part of the proposed statistical framework has been presented in this chapter.
The goal of such an approximation was to evaluate the output of a computer simulation code for an arbitrary input much faster.
The approximation is based on Gaussian stochastic process resulting in a statistical metamodel.
As the dimensionality of the output is large, in the order of tens of thousands, a dimension reduction step is adopted by means of \gls[hyper=false]{pca} (an approach similar to what was adopted in Chapter 3).

The results obtained on the \gls[hyper=false]{trace} model \gls[hyper=false]{feba} is reasonable.
Though the prediction error can at times be large, the metamodel gives an overall good performance on average and in context for the three types of multivariate output (clad temperature, pressure drop, and liquid carryover). 
The limitation of the approach is mainly for the output which exhibits strong non-linearity and discontinuity (such as the quenching in the clad temperature transient).
This, in turn, is due to the use of \gls[hyper=false]{pca} as the (linear) dimension reduction tool.
As such, a first step of improvement in this regard can be aimed toward replacing \gls[hyper=false]{pca} with another, more advanced dimension reduction tool.

Using the \gls[hyper=false]{gp} \gls[hyper=false]{pc} metamodel as the surrogate for \gls[hyper=false]{trace} run, 
the prediction for arbitrary model parameters values can be made much faster ($< 5 [s]$ per metamodel evaluation vs. $6-15 \, [min]$ per \gls[hyper=false]{trace} run).
As such the metamodel constructed in this chapter can be used as the basis for Bayesian model calibration which requires tens if not hundreds of thousands function evaluations. 
However, it is also important to note that the time required for the construction of the metamodel and as well as for its convergence study has to be taken into account.
The training, validation, and testing data have to be generated from actual code runs.
Additionally, the model fitting step to estimate \gls[hyper=false]{gp} metamodel hyper-parameters is an optimization problem that can easily become expensive for large training samples of large dimension (large number of input parameters).
 
The study also confirms that the size of the training sample is the main factor in determining the predictive performance of the metamodel.
The choice of covariance function has some impact especially in relation to the stability of the performance,
while the choice of experimental design has a neglible impact on the performance.

%*****************************************
%*****************************************
%*****************************************
%*****************************************
%*****************************************
