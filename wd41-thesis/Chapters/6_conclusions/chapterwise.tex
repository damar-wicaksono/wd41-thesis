%****************************************************************
\section{Chapter-wise Summary}\label{sec:conclusions_chapterwise}
%****************************************************************

% Chapter 1
Chapter~\ref{ch:intro} introduced the doctoral research by first introducing the problem of uncertainty quantification in nuclear engineering \gls[hyper=false]{th} analysis; both as a forward and a backward (inverse) problems. 
The generic problem of inverse uncertainty quantification was then put in the context of the recently concluded \gls[hyper=false]{premium} project; a benchmark project comparing different inverse uncertainty quantification methods used by the community. 
The chapter then presented a set of strategies proposed in this thesis to quantify the uncertainty, namely sensitivity analysis, statistical metamodeling, and Bayesian calibration. 
The set of strategies was a consolidated statistical framework adapted from the applied statistical literature, on which a review was then presented.

% Chapter 2
Chapter~\ref{ch:trace_reflood} presented the reflood experiment at the \glsentryshort{feba} facility that served as the experimental basis of this work.
The \gls[hyper=false]{trace} model of the facility was developed and a set of $27$ initial input parameters perceived to be important for the simulation was selected.
Thereafter, prior uncertainties of the selected input parameters were assigned and they were propagated through the \gls[hyper=false]{trace} model of \gls[hyper=false]{feba} to assess the prior level of prediction uncertainties.
This model then became the running case study in the three subsequent chapters to which the proposed methods are applied.

% Chapter 3
Chapter~\ref{ch:gsa} introduced selected \gls[hyper=false]{gsa} methods which were then applied to the \gls[hyper=false]{trace} model of \gls[hyper=false]{feba}.
First, the importance of the initial set of input parameters was quantitatively assessed through the Morris screening method and the Total-effect Sobol indices.
The two provided a basis for parameter screening in which less influential parameters were excluded from further analysis, reducing the size of the problem.
After the screening step, only $12$ out of the initial $27$ input parameters were found to be influential. 
Then, focusing on the $12$ most influential parameters, the effect of parameter perturbation on the overall time-dependent outputs was investigated.
The high-dimensionality of the outputs was reduced by means of techniques derived from \gls[hyper=false]{fda}.
Finally, main- and total effect Sobol' indices, two global sensitivity measures, were estimated for each parameter with respect to the output in the reduced space.
The results regarding parameters sensitivity with respect to different outputs have provided a better understanding on the inputs/outputs relationship in the \gls[hyper=false]{trace} model of \gls[hyper=false]{feba}.
By analyzing how the model actually behaved (or should have behaved) under parameters perturbation has also provided an additional step of the model verification and validation.

% Chapter 4
Chapter~\ref{ch:gp_metamodel} detailed the development of metamodel based on \gls[hyper=false]{gp} to substitute \gls[hyper=false]{trace} runs.

% Chapter 5
