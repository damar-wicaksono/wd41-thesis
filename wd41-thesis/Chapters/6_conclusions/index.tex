%***************************************************************************************
\chapter[Conclusions and Future Work]{Conclusions and Future Work}\label{ch:conclusions}
%***************************************************************************************

% Opening Paragraph
The main goal of the present doctoral research was to quantify the uncertainty of physical model parameters implemented in \gls[hyper=false]{th} system codes.
To that end, a methodology has been developed -- and its application demonstrated -- to quantify the uncertainty of physical model parameters important in the simulation of reflood experiment; relevant phenomena to consider in the safety analysis of \glspl[hyper=false]{lwr}.
The methodology consisted of three statistical methods for \emph{sensitivity analysis}, \emph{metamodeling}, and \emph{Bayesian calibration}.

Starting from the \gls[hyper=false]{trace} code modeling of the \gls[hyper=false]{feba} facility for reflood experiment and the initial selection of uncertain input parameters,
sensitivity analysis methods were then applied in essence to assess the importance of each selected input parameters and to select model parameters that were truly important for the reflood simulation.
In anticipation of a high computational cost associated with the Bayesian calibration, a statistical metamodel of the \gls[hyper=false]{trace} model of \gls[hyper=false]{feba} based on \glsfirst[hyper=false]{gp} was then developed and validated.
Then, using the validated metamodel to substitute \gls[hyper=false]{trace} code run, the selected model parameters were calibrated against the experimental data of \gls[hyper=false]{feba} which resulted in an a posteriori quantification of the parameters uncertainties.
Finally, the quantified uncertainties were verified by means of uncertainty propagation on \gls[hyper=false]{feba} tests with boundary conditions different from the conditions of the calibration data.

% What is this chapter all about
This final chapter starts with a chapter-wise summary of the thesis, presented in Section~\ref{sec:conclusions_chapterwise}.
The main achievements of the thesis are given in Section~\ref{sec:conclusions_achievements}, while recommendations for future work are proposed in Section~\ref{sec:conclusions_recommendations}.
The chapter -- and the thesis -- is finally concluded in Section~\ref{sec:conclusions_concluding_remarks}.

% The contents of the chapter
%****************************************************************
\section{Chapter-wise Summary}\label{sec:conclusions_chapterwise}
%****************************************************************

% Chapter 1
Chapter~\ref{ch:intro} introduced the doctoral research through the problem of uncertainty quantification in nuclear engineering \gls[hyper=false]{th} analysis; both as a forward and a backward (inverse) problems. 
The particular problem of inverse uncertainty quantification was then put in the context of the recently concluded OECD/NEA \gls[hyper=false]{premium} project;
a benchmark project comparing different inverse uncertainty quantification methods used by the community. 
The chapter then presented a set of strategies proposed in this thesis to quantify the uncertainty, namely sensitivity analysis, statistical metamodeling, and Bayesian calibration. 
The set of strategies was a consolidated statistical framework adapted from the applied statistical literature, on which a review was conducted.

% Chapter 2
Chapter~\ref{ch:trace_reflood} presented the reflood experiment at the \glsentryshort{feba} facility that served as the experimental basis of this work.
The \gls[hyper=false]{trace} model of the facility was developed and a set of $27$ initial input parameters perceived to be important for the simulation was selected.
Thereafter, prior uncertainties of the selected input parameters were assigned and they were propagated through the \gls[hyper=false]{trace} model of \gls[hyper=false]{feba} to assess the prior level of prediction uncertainties.
This model then became the running case study in the three subsequent chapters to which the proposed methods are applied.

% Chapter 3
Chapter~\ref{ch:gsa} introduced selected \gls[hyper=false]{gsa} methods which were then applied to the \gls[hyper=false]{trace} model of \gls[hyper=false]{feba}.
First, the importance of the initial set of input parameters was quantitatively assessed through the Morris screening method and the Total-effect Sobol indices.
The two provided a basis for parameter screening in which less influential parameters were excluded from further analysis, reducing the size of the problem.
After the screening step, only $12$ out of the initial $27$ input parameters were found to be influential. 
Then, focusing on the $12$ most influential parameters, the effect of parameter perturbation on the overall time-dependent outputs was investigated.
The high-dimensionality of the outputs was reduced by means of techniques derived from \gls[hyper=false]{fda}.
Finally, main- and total-effect Sobol' indices, two global sensitivity measures, were estimated for each parameter with respect to the output in the reduced space.
The results regarding parameters sensitivity with respect to different outputs have provided a better understanding on the inputs/outputs relationship in the \gls[hyper=false]{trace} model of \gls[hyper=false]{feba}.
By analyzing how the model actually behaved (or should have behaved) under parameters perturbation has also provided an additional step into the model verification and validation.

% Chapter 4
Chapter~\ref{ch:gp_metamodel} detailed the development and validation of a metamodel based on \gls[hyper=false]{gp} to substitute the \gls[hyper=false]{trace} model \gls[hyper=false]{feba}.
Though a single run of the \gls[hyper=false]{trace} model was relatively short ($\approx 6-14\,[min]$), a large number of runs in the order of hundreds of thousands was expected for the Bayesian calibration.  
Thus, a computationally efficient metamodel was deemed crucial in the calibration of the model parameters.
Built upon the results of the previous chapter, the development was directly focused on the $12$ most influential input parameters.
The high dimensionality of the output, in time and in space, was dealt with \gls[hyper=false]{fda}, a linear dimension reduction method.
The dimension reduction method was shown to have difficulty in representing the cladding temperature output which exhibited strong discontinuity in the vicinity of quenching. 
Yet, the average predictive performance of the metamodel against a test dataset of actual \gls[hyper=false]{trace} runs was found to be acceptable, especially in comparison to the initial prediction uncertainty due to the prior input parameters uncertainties.
The validated metamodel, with a cost of $< 5\,[s]$ per evaluation, was then ready to be used over the prior range of the input parameters in lieu of directly running \gls[hyper=false]{trace}.

% Chapter 5
Chapter~\ref{ch:bayesian_calibration}, the last of the main chapters of the thesis, finally proceeded with the a posteriori quantification of uncertainties of the most influential reflood model parameters on the basis of \gls[hyper=false]{feba} test No. $216$.
Different posteriors \glspl[hyper=false]{pdf} corresponding to different calibration schemes were formulated and directly simulated using an \glsentryshort{aies} ensemble \glsentryshort{MCMC} sampler.
Five different calibration having different assumptions were investigated: with or without considering model bias term, incorporating different types of data, and including or excluding a strongly correlated model parameter.  
Two types of parameter non-identifiability were encountered: parameter non-identifiability due to insensitivity with respect to a type of data and non-identifiability due to correlation between parameters.
The former was solved by considering different types of output that the parameter of interest was sensitive to.
The latter was more challenging as considering different types of data did not manage to solve the non-identifiability issue (the univariate marginals of the parameters remained large). 
At the same time, while excluding one of the correlated parameters did allow for a more precise estimation of the other parameters that were previously correlated,
the prior uncertainty of the excluded parameter kept the prediction uncertainty band relatively wider.
However, even without precise estimates of each parameter, the correlation structure among model parameters provided a set of ``collective-fitted'' values that was consistent with the calibration data.
Specifically, as long as the correlation structure was kept, propagation with parameters with large univariate marginal uncertainties would still produce prediction that was consistent with the data. 
  
The calibration scheme with considering model bias term and incorporating all types of outputs was able to constrain the prior uncertainties of the model parameters while keeping the nominal \gls[hyper=false]{trace} parameters values within the posterior uncertainty interval.
That was in contrast with the results of the calibration scheme without considering model bias term, in which the posterior uncertainties were concentrated on either or both sides of the prior range, and at times having the nominal \gls[hyper=false]{trace} parameters values outside the posterior uncertainty interval. 
However, the performance of these two posteriors were found to be similar across \gls[hyper=false]{feba} tests with the calibration scheme without considering model bias was slightly more informative (having tighter uncertainty band) but less calibrated (having more experimental data points outside the uncertainty band) compared to the performance of the calibration scheme with considering model bias. 
Furthermore, except for a few outputs (namely the cladding temperature output at the top assembly and the liquid carryover), the relative performance of all posterior uncertainties was insensitive to boundary conditions of the different \gls[hyper=false]{feba} tests.
%************************************************************************************************************************************
\section[Achievements and Recommendations]{Main Achievements and Recommendations for Future Work}\label{sec:conclusions_achievements}
%************************************************************************************************************************************

% Main Objectives Revisited
The thesis proposed the application of a set of methods adapted from the applied literature with the ultimate goal to quantify the uncertainty of model parameters in a \gls[hyper=false]{th} system code.
The application of each method was illustrated and demonstrated on the basis of a reflood experiment simulation model in the \gls[hyper=false]{trace} code.
According to Section~\ref{sub:intro_objectives} the listed objectives of the proposed methods were to:
\begin{itemize}
	\item analyze and better understand the inputs/outputs relationship in a computer simulation with uncertain input;
	\item approximate the inputs/outputs relationship of a complex computer simulation for a faster evaluation; and,
	\item calibrate the physical model parameters against various relevant experimental data.
\end{itemize}

% What was done
During the course of this doctoral research, each of these methods were investigated and it was applied to the running example of the \gls[hyper=false]{feba} reflood facility simulation model in the \gls[hyper=false]{trace} code. 
Each of these applications was aimed to illustrate the particularities -- and difficulties -- of applying the method the \gls[hyper=false]{trace} model as well as to demonstrate the values of the method.
Chapters~\ref{ch:gsa}--\ref{ch:bayesian_calibration} provided a detailed account on the methods and their applications, of which the main achievements are highlighted below.
Given the limited scope and duration of the project, many difficulties found along the way remained unaddressed, and they are the basis for the proposed recommendations.

% PREMIUM
The thesis project was initiated by the participation of \glsentryshort{lrs} at \gls[hyper=false]{psi} in the \gls[hyper=false]{oecd}/\gls[hyper=false]{nea} \gls[hyper=false]{premium} benchmark.
The work related to that participation also constitutes a portion -- and achievement -- of the thesis project.

% Publications
Finally, four papers were presented in international conferences \cite{Wicaksono2014,Wicaksono2014a,Wicaksono2015,Wicaksono2016}, a journal article was published \cite{Wicaksono2016b}, and two contributions were submitted \cite{Wicaksono2016a,Zerkak2016} to the \gls[hyper=false]{premium} project and included in the NEA reports \cite{Reventos2016,Sanz2017}.

%-----------------------------------------------------
\subsection{Contributions to OECD/NEA PREMIUM Project}
%-----------------------------------------------------

% Contributions to PREMIUM
The work related to the contribution to the \gls[hyper=false]{premium} project comprises the bulk of Chapter~\ref{ch:trace_reflood}.
The \gls[hyper=false]{trace} model of \gls[hyper=false]{feba} was successfully developed within that context and became the basis for several follow-up studies.
\marginpar{TRACE model of FEBA}
The model was stable and was relatively quick to run allowing even a relatively brute force sensitivity analysis method to be applied.
It is now part of the in-house \gls[hyper=false]{trace} code validation database at \glsentryshort{lrs}.

% Prior quantification
The prior uncertainties of the input parameters were quantified under a close supervision by thesis supervisor at \glsentryshort{psi} \cite{Zerkak2016}.
\marginpar{Contribution to PREMIUM, prior uncertainty quantification}
The quantified uncertainties were then propagated both on the \gls[hyper=false]{trace} models of \gls[hyper=false]{feba} and PERICLES (another reflood facility not presented in this thesis).
The result of the propagation submitted to the \gls[hyper=false]{premium} project were deemed satisfactory as it served the purpose of the prior quantification.
That is, the prediction uncertainties of both facilities were wide but covered the experimental data well, confirming that the prior range was not underestimated.

% trace simexp
Still within the context of \gls[hyper=false]{premium}, a python scripting tool was developed to assist in conducting computer experiment on the \gls[hyper=false]{trace} model of \gls[hyper=false]{feba}.
\marginpar{\texttt{trace-simexp}}
The tool \texttt{trace-simexp} has reached a stable version, is well documented, and has been applied for several follow-up studies within and without the scope of the present doctoral research.

\paragraph{Recommendations for Future Work}\mbox{}\\

Although stable, the current version of \texttt{trace-simexp} has been only tested so far for the \gls[hyper=false]{trace} model of \gls[hyper=false]{feba}.
Extension to other \gls[hyper=false]{trace} models are feasible.
However, depending on the complexity of those models, further development of the tool might become unrealistic and it would be better to opt for the use of an integrated uncertainty framework 
(e.g., \texttt{UQLab}, \texttt{Dakota}, \texttt{OpenTurns}, \texttt{Uranie}).
Typically, such framework supports an application programming interface (API) to a make connection with an external simulation model or to a third-party program.
It does require initial effort of getting acquainted with the terminologies of each framework, some are easier than the other, but in the long run for a generic complex model they might be the solution.

%--------------------------------------------------------------------------------------------------------------------------------------------------------------------------
\subsection{Implementation and application of GSA methods (to analyze and better understand the inputs/outputs relationship in a computer simulation with uncertain input)}
%--------------------------------------------------------------------------------------------------------------------------------------------------------------------------

% Sensitivity Analysis, Morris
Three conference papers \cite{Wicaksono2014,Wicaksono2014a,Wicaksono2015} and a journal article \cite{Wicaksono2016b} made up Chapter~\ref{ch:gsa}.
The size of the initial selection of input parameters, as exemplified in \gls[hyper=false]{premium}, can be large. 
\marginpar{Implementation and application of screening methods}
Lacking prior knowledge, the selection should also include all the parameters that are vaguely perceived as important. 
The implementation and the application of screening methods (Morris screening method and Sobol' total-effect indices), as demonstrated in this thesis for the \gls[hyper=false]{trace} model of \gls[hyper=false]{feba},
allows for a quick, systematic, and quantitative screening of the initial set of input parameters in a global manner (i.e. simultaneous perturbation over the whole range of parameter uncertainties).
In the case studied here, more than half of the initial selection were found to be non-influential to the reflood simulation.

% FDA
In conjunction with that, \gls[hyper=false]{fda} techniques to characterize the variation of functional data set was investigated and successfully applied to analyze the variation in reflood curves.
In essence, the application of the \gls[hyper=false]{fda} techniques resulted in the representation of the time-dependent output in a reduced space.
Indeed, time- and space-dependent outputs are ubiquitous in the \gls[hyper=false]{th} analysis thus dimension reduction techniques are worth investigating.
\marginpar{Application of GSA coupled with FDA}
\gls[hyper=false]{gsa} methods and \gls[hyper=false]{fda} techniques were then coupled together to decompose the variance of the output in the reduced space.
This was done through the implementation and application of \gls[hyper=false]{mc} methods to estimate the Sobol', main- and total-effect, indices.  
The sensitivity analysis reveals interesting behavior of the \gls[hyper=false]{trace} model of \gls[hyper=false]{feba} in terms of interactions

% GSA
The implementations of the employed \gls[hyper=false]{gsa} methods were developed in-house as a python module to allow full internal control.
\marginpar{\texttt{gsa-module}}
The module \texttt{gsa-module} is documented and was tested against a suite of test functions and applied to obtain all the results presented in Chapter~\ref{ch:gsa}.

\paragraph{Recommendations for Future Work}\mbox{}\\

% screening methods

% sobol decomposition
 
% gsa module and uncertainty framework

%---------------------------------------------------------------------------------------------------------------------------------------------------------------------
\subsection{Development and validation of a TRACE metamodel (to approximate the inputs/outputs relationship of a complex computer simulation for a faster evaluation)}
%---------------------------------------------------------------------------------------------------------------------------------------------------------------------

% Metamodeling and Bayesian
\glsfirst[hyper=false]{gp} metamodeling was demonstrated for the \gls[hyper=false]{trace} model of \gls[hyper=false]{feba} having high-dimensional outputs.
In this thesis, the high-dimensionality of the outputs was treated by \gls[hyper=false]{pca} resulting in a \gls[hyper=false]{gp} \gls[hyper=false]{pc} metamodel.
The validation and testing steps then showed that the error of the metamodel across the prior range of input parameters were within a reasonable range.
In other words, it managed to approximate the important features of the inputs/outputs relationship of the reflood simulation model in \gls[hyper=false]{trace}.
Using the \glsfirst[hyper=false]{gp} \glsfirst[hyper=false]{pc} as the surrogate for TRACE run, the prediction for arbitrary input parameters values could be made much faster
(i.e., $< 5\,[s]$ per metamodel evaluation vs. $6 - 15\,[min]$ per TRACE).

The thesis has demonstrated the applicability of a linear dimension reduction technique to reduce the high dimension of the output.
The technique performed best for relatively smooth output (in this particular application, the pressure drop and liquid carryover transients), while it performed worse for reconstructing an output exhibiting strong discontinuity (i.e., the cladding temperature output exhibited a discontinuity around quenching).
Finally, though many practical aspects were involved in the construction of the metamodel, the work in the thesis concluded that the size of training samples (i.e., the actual code runs) was the most important factor;
if they can be afforded, more runs should be conducted.

\paragraph{Recommendations for Future Work}\mbox{}\\

The worse performance of the \gls[hyper=false]{pca} on reconstructing the cladding temperature output was, in turn, due to the use of \gls[hyper=false]{pca} as the linear dimension reduction.
\marginpar{Alternative dimension reduction technique}
Furthermore, the size of training size might get inflated because of this worse performance:
large training samples was required not because the inputs/outputs relationship was particularly complex, but because reconstruction of the output with small error required more samples. 
As such, a first step of improvement in this regard can be aimed toward replacing PCA with another, more advanced dimension reduction tool.
Simulations with high-dimensional outputs, either in time or space, are typical in \gls[hyper=false]{th} analysis.
It is thus worth investigating the application of different dimension reduction techniques, linear (extension of \gls[hyper=false]{pca}, e.g., \cite{Zhang2005}) or nonlinear (e.g., isomap \cite{Tenenbaum2000} and locally linear embedding (LLE) \cite{Roweis2000}).
Many of such developments are made in the area of image processing.
Indeed as shown in Chapter~\ref{ch:gp_metamodel}, a $1$-dimensional time-dependent \gls[hyper=false]{trace} simulation output can be represented as an image.

Furthermore, \gls[hyper=false]{gp} metamodel is not the only available metamodeling technique.
\marginpar{Alternative metamodeling techniques}
The response surface method was traditionally employed for \gls[hyper=false]{th} system analysis but more advanced techniques are currently available such as the ones mentioned in Section~\ref{sub:intro_statistical_metamodeling}. 
The investigation on their applicability -- the predictive performance and the computational cost of construction -- for a variety of \gls[hyper=false]{th} models is of interest in its own right.

Finally, the step proposed in this thesis is to conduct sensitivity analysis before to construct the metamodel.
\marginpar{Alternative workflow}
In that case, metamodeling error can be excluded from the sensitivity analysis.
However, it is also possible to construct the metamodel in advanced before moving on to the sensitivity analysis step.
Some of metamodeling techniques allows the metamodeling and sensitivity analysis to be combined while providing an estimate of the associated error.
In particular \gls[hyper=false]{pce}, allows the computation of Sobol' sensitivity indices to be computed by post-processing the resulting coefficients of the expansion \cite{Sudret2008}.

%-----------------------------------------------------------------------------------------------------------------
\subsection{Bayesian calibration of the TRACE reflood model parameters against various relevant experimental data}
%-----------------------------------------------------------------------------------------------------------------

% Main Achievements

\paragraph{Recommendations for Future Work}\mbox{}\\
%*******************************************************************************
\section{Recommendations for Future Work}\label{sec:conclusions_recommendations}
%*******************************************************************************

% Use of the Metamodel

% parameter Interaction

% Advanced use of FDA and Non-linear dimension reduction

% Incorporating more Data and Hierarchical Modeling

\input{Chapters/6_conclusions/concluding_remarks}
%************************************************