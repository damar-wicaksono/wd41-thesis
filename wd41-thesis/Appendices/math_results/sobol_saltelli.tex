\section{The Sobol'-Saltelli Method for Estimating Variance-Based Sensitivity Indices}\label{app:sobol_saltelli}

Sobol' \cite{Sobol2001} and Saltelli \cite{Saltelli2002} proposed an alternative approach that circumvent the nested structure of \glsfirst[hyper=false]{mc} simulation to estimate sensitivity indices (see Algorithm~\ref{alg:brute_force_mc}).
The formulation starts by expressing the expectation and variance operators in their integral form.
As the following formulation is defined on a unit hypercube of $D$-dimension input parameter space where each parameter is a uniform and independent random variable,
explicit writing of the distribution within the integration as well as the integration range are excluded for conciseness.

First, the variance operator shown in the numerator of Eq.~(\ref{eq:sa_main_effect_index}) is written as
\begin{equation}
  \begin{split}
    \mathbb{V}_{d}[\mathbb{E}_{\sim d}[\mathcal{Y}|\mathcal{X}_d]] & = \mathbb{E}_{d}[\mathbb{E}_{\sim d}^{2}[\mathcal{Y}|\mathcal{X}_d]] - \left(\mathbb{E}_{d}[\mathbb{E}_{\sim d}[\mathcal{Y}|\mathcal{X}_d]]\right)^2 \\ 
                                               & = \int \mathbb{E}_{\sim d}^{2}[\mathcal{Y}|\mathcal{X}_d] dx_d - \left(\int \mathbb{E}_{\sim d}[\mathcal{Y}|\mathcal{X}_d] dx_d\right)^2
  \end{split}
\label{eq:ss_variance_integral}
\end{equation}
The notation $\mathbb{E}_{\sim \circ}[\circ | \circ]$ was already explained in Section~\ref{sub:sa_hdmr}, 
while $\mathbb{E}_{\circ} [\circ]$ corresponds to the marginal expectation operator 
where the integration is carried out over the range of parameters specified in the subscript. 

Next, consider the term conditional expectation shown in Eq.~(\ref{eq:ss_variance_integral}), which per definition reads
\begin{equation}
  \mathbb{E}_{\sim d} [\mathcal{Y}|\mathcal{X}_d] = \int f(\bm{x}_{\sim d}, x_d) d\bm{x}_{\sim d}
\label{eq:ss_expectation_integral}
\end{equation}
Note that $\bm{x} = \{\bm{x}_{\sim d}, x_d\}$, such that
\begin{equation}
  \int \mathbb{E}_{\sim d} [\mathcal{Y}|\mathcal{X}_d] dx_d = \int \int f(\bm{x}_{\sim d}, x_d) d\bm{x}_{\sim d} dx_{d} = \int f(\bm{x}) d\bm{x}
\label{eq:ss_expectation_expectation}
\end{equation}

Following the first term of Eq.~(\ref{eq:ss_variance_integral}),
by squaring Eq.~(\ref{eq:ss_expectation_integral})
and by defining a dummy vector variable $\bm{x}^{\prime}_{\sim d}$, 
the product of the two integrals can be written in terms of a single multiple integrals
\begin{equation}
  \begin{split}
    \mathbb{E}_{\sim d}^{2} [\mathcal{Y}|\mathcal{X}_d] & = \int f(\bm{x}_{\sim d}, x_d) d\bm{x}_{\sim d} \cdot \int f(\bm{x}_{\sim d}, x_d) d\bm{x}_{\sim d} \\
                                    & = \int \int f(\bm{x}^{\prime}_{\sim d}, x_d) f(\bm{x}_{\sim d}, x_d) d\bm{x}^{\prime}_{\sim d} d\bm{x}_{\sim d}
  \end{split}
\label{eq:ss_multiple_integrals}
\end{equation}

Returning to the full definition of variance of conditional expectation in Eq.~(\ref{eq:ss_variance_integral}),
\begin{equation}
  \begin{split}
    \mathbb{V}_{d}[\mathbb{E}_{\sim d}[\mathcal{Y}|\mathcal{X}_d]] & = \int \int f(\bm{x}^{\prime}_{\sim d}, x_d) f(\bm{x}_{\sim d}, x_d) d\bm{x}^{\prime}_{\sim d} d\bm{x}_{\sim d} \\
                                               & \quad - \left(\int f(\bm{x}) d\bm{x}\right)^2
  \end{split}
\label{eq:ss_variance_integral_single}
\end{equation}

Finally, the main-effect sensitivity index can be written as an integral as follows:
\begin{equation}
  \begin{split}
    S_d & = \frac{\mathbb{V}_d [\mathbb{E}_{\sim d} [\mathcal{Y}|\mathcal{X}_d]]}{\mathbb{V}[\mathcal{Y}]} \\
        & = \frac{\int \int f(\bm{x}^{\prime}_{\sim d}, x_d) f(\bm{x}_{\sim d}, x_d) d\bm{x}^{\prime}_{\sim d} d\bm{x} - \left(\int f(\bm{x}) d\bm{x}\right)^2}{\int f(\bm{x})^2 d\bm{x} - \left( \int f(\bm{x}) d\bm{x}\right)^2}
  \end{split}
\label{eq:ss_main_effect_integral}
\end{equation}
The integral form given above dispenses with the nested structure of multiple integrals in the original definition of main-effect index.
The multidimensional integration is over $(2 \times D - 1)$-dimension
and it is the basis of estimating sensitivity index using \glsfirst[hyper=false]{mc} simulation in this thesis, hereinafter referred to as the Sobol'-Saltelli method.
The same procedure applies to derive the total-effect index which yields,
\begin{equation}
  \begin{split}
    ST_d & = \frac{\mathbb{E}_{\sim d}[\mathbb{V}_{d}[\mathcal{Y}|\bm{\mathcal{X}}_{\sim d}]]}{\mathbb{V}[\mathcal{Y}]} \\
        & = \frac{\int f^2(\bm{x}) d\bm{x} - \int \int f(\bm{x}_{\sim d}, x^{\prime}_d) f(\bm{x}_{\sim d}, x_d) d\bm{x}^{\prime}_{d} d\bm{x}}{\int f^2(\bm{x}) d\bm{x} - \left( \int f(\bm{x}) d\bm{x}\right)^2}
  \end{split}
\label{eq:ss_total_effect_integral}
\end{equation}